%===============================================================================
% LaTeX sjabloon voor de bachelorproef toegepaste informatica aan HOGENT
% Meer info op https://github.com/HoGentTIN/latex-hogent-report
%===============================================================================

\documentclass[dutch,dit,thesis]{hogentreport}

% TODO:
% - If necessary, replace the option `dit`' with your own department!
%   Valid entries are dbo, dbt, dgz, dit, dlo, dog, dsa, soa
% - If you write your thesis in English (remark: only possible after getting
%   explicit approval!), remove the option "dutch," or replace with "english".

\usepackage{lipsum} % For blind text, can be removed after adding actual content

%% Pictures to include in the text can be put in the graphics/ folder
\graphicspath{{../graphics/}}

%% For source code highlighting, requires pygments to be installed
%% Compile with the -shell-escape flag!
%% \usepackage[chapter]{minted}
%% If you compile with the make_thesis.{bat,sh} script, use the following
%% import instead:
\usepackage[chapter]{minted}
\usemintedstyle{solarized-light}

%% Formatting for minted environments.
\setminted{%
    autogobble,
    frame=lines,
    breaklines,
    linenos,
    tabsize=4
}

%% Ensure the list of listings is in the table of contents
\renewcommand\listoflistingscaption{%
    \IfLanguageName{dutch}{Lijst van codefragmenten}{List of listings}
}
\renewcommand\listingscaption{%
    \IfLanguageName{dutch}{Codefragment}{Listing}
}
\renewcommand*\listoflistings{%
    \cleardoublepage\phantomsection\addcontentsline{toc}{chapter}{\listoflistingscaption}%
    \listof{listing}{\listoflistingscaption}%
}

% Other packages not already included can be imported here

%%---------- Document metadata -------------------------------------------------
% TODO: Replace this with your own information
\author{Stein Van Driessche}
\supervisor{Dhr. B. Van Vreckem}
\cosupervisor{Dhr. T. De Quick}
\title[Vertrouwelijk]%
    {Ontwerp van een Veilige en Flexibele Synchronisatie- en Back-upstrategie voor Electronic Flight Bags in Operationele Contexten}
\academicyear{\advance\year by -1 \the\year--\advance\year by 1 \the\year}
\examperiod{2}
\degreesought{\IfLanguageName{dutch}{Professionele bachelor in de toegepaste informatica}{Bachelor of applied computer science}}
\partialthesis{false} %% To display 'in partial fulfilment'
%\institution{Internshipcompany BVBA.}

%% Add global exceptions to the hyphenation here
\hyphenation{back-slash}

%% The bibliography (style and settings are  found in hogentthesis.cls)
\addbibresource{bachproef.bib}            %% Bibliography file
\addbibresource{../voorstel/voorstel.bib} %% Bibliography research proposal
\defbibheading{bibempty}{}

%% Prevent empty pages for right-handed chapter starts in twoside mode
\renewcommand{\cleardoublepage}{\clearpage}

\renewcommand{\arraystretch}{1.2}

%% Content starts here.
\begin{document}

%---------- Front matter -------------------------------------------------------

\frontmatter

\hypersetup{pageanchor=false} %% Disable page numbering references
%% Render a Dutch outer title page if the main language is English
\IfLanguageName{english}{%
    %% If necessary, information can be changed here
    \degreesought{Professionele Bachelor toegepaste informatica}%
    \begin{otherlanguage}{dutch}%
       \maketitle%
    \end{otherlanguage}%
}{}

%% Generates title page content
\maketitle
\hypersetup{pageanchor=true}

\input{voorwoord}
\input{samenvatting}

%---------- Inhoud, lijst figuren, ... -----------------------------------------

\tableofcontents

% In a list of figures, the complete caption will be included. To prevent this,
% ALWAYS add a short description in the caption!
%
%  \caption[short description]{elaborate description}
%
% If you do, only the short description will be used in the list of figures

\listoffigures

% If you included tables and/or source code listings, uncomment the appropriate
% lines.
\listoftables

\listoflistings

% Als je een lijst van afkortingen of termen wil toevoegen, dan hoort die
% hier thuis. Gebruik bijvoorbeeld de ``glossaries'' package.
% https://www.overleaf.com/learn/latex/Glossaries

%---------- Kern ---------------------------------------------------------------

\mainmatter{}

% De eerste hoofdstukken van een bachelorproef zijn meestal een inleiding op
% het onderwerp, literatuurstudie en verantwoording methodologie.
% Aarzel niet om een meer beschrijvende titel aan deze hoofdstukken te geven of
% om bijvoorbeeld de inleiding en/of stand van zaken over meerdere hoofdstukken
% te verspreiden!

%%=============================================================================
%% Inleiding
%%=============================================================================

\chapter{\IfLanguageName{dutch}{Inleiding}{Introduction}}%
\label{ch:inleiding}

In moderne luchtvaartomgevingen spelen Electronic Flight Bags (EFB's) een essentiële rol in het ondersteunen van piloten bij het uitvoeren van vluchten. Deze tablets vervangen papieren handleidingen en vluchtplannen en bieden toegang tot kritieke informatie zoals navigatiekaarten, weersvoorspellingen en operationele richtlijnen. Het gebruik van EFB's brengt echter specifieke uitdagingen met zich mee, vooral in operationele settings waar de beschikbaarheid van internet beperkt of geheel afwezig is, zoals in afgelegen regio's of tijdelijke locaties. Dit bemoeilijkt betrouwbare synchronisatie en back-ups van essentiële data, wat een risico vormt voor operationele continuïteit.

\section{\IfLanguageName{dutch}{Probleemstelling}{Problem Statement}}%
\label{sec:probleemstelling}

Een concreet probleem waarmee de Belgische Luchtmacht in deze context wordt geconfronteerd, is hoe zij gevoelige data op EFB's kunnen beveiligen en synchroniseren zonder afhankelijk te zijn van constante internettoegang. Momenteel worden synchronisatieprocessen vaak uitgevoerd via clouddiensten zoals Nextcloud, maar deze oplossingen voldoen niet altijd aan de eisen voor veilige back-ups en operationele betrouwbaarheid. Daarnaast brengt het werken in onveilige netwerkomgevingen, zoals openbare wifi in hotels of lokale computers op geïmproviseerde locaties, extra risico’s met zich mee voor datalekken en gegevensverlies. Het ontbreken van een robuuste en gestandaardiseerde methode vergroot deze kwetsbaarheden en kan de operationele continuïteit in gevaar brengen.

\section{\IfLanguageName{dutch}{Onderzoeksvraag}{Research question}}%
\label{sec:onderzoeksvraag}

Dit onderzoek richt zich op het ontwikkelen van een oplossing voor deze uitdagingen en wordt gedreven door de volgende centrale onderzoeksvraag:

\textbf{Hoe kan een veilige en gecontroleerde omgeving worden ontworpen voor de synchronisatie en back-up van data in Electronic Flight Bags zonder of met beperkte internetverbinding, om gegevensverlies te minimaliseren?}

Om deze hoofdvraag te beantwoorden, worden diverse deelvragen geformuleerd. Deze richten zich zowel op het identificeren van de problemen als op het ontwikkelen van mogelijke oplossingen:

\begin{itemize}
    \item Welke veiligheidsrisico’s en technische beperkingen zijn inherent aan de huidige methoden voor synchronisatie en back-ups?
    \item Welke technologieën en procedures kunnen bijdragen aan een veilige en efficiënte gegevensoverdracht in afwezigheid van internet?
\end{itemize}

Door deze vragen te beantwoorden, kan het onderzoek zowel de bestaande uitdagingen analyseren als concrete oplossingen voorstellen.

\section{\IfLanguageName{dutch}{Onderzoeksdoelstelling}{Research objective}}%
\label{sec:onderzoeksdoelstelling}

Het doel van dit onderzoek is om een concrete, schaalbare oplossing te ontwikkelen in de vorm van een proof-of-concept (PoC). Deze oplossing zal verschillende scenario’s simuleren, waaronder data-overdracht via lokale netwerken, het gebruik van veilige bestandsoverdrachten en de integratie van back-uptechnologieën zoals Veeam. Hierbij worden belangrijke operationele parameters zoals Recovery Time Objectives (RTO) en Recovery Point Objectives (RPO) in kaart gebracht. Dit onderzoek beoogt niet alleen een praktische oplossing te bieden, maar draagt ook bij aan de ontwikkeling van richtlijnen en best practices voor het gebruik van EFB's in complexe operationele contexten.


\section{\IfLanguageName{dutch}{Opzet van deze bachelorproef}{Structure of this bachelor thesis}}%
\label{sec:opzet-bachelorproef}

% Het is gebruikelijk aan het einde van de inleiding een overzicht te
% geven van de opbouw van de rest van de tekst. Deze sectie bevat al een aanzet
% die je kan aanvullen/aanpassen in functie van je eigen tekst.

De rest van deze bachelorproef is als volgt opgebouwd:

In Hoofdstuk~\ref{ch:stand-van-zaken} wordt een overzicht gegeven van de stand van zaken binnen het onderzoeksdomein, op basis van een literatuurstudie.

In Hoofdstuk~\ref{ch:methodologie} wordt de methodologie toegelicht en worden de gebruikte onderzoekstechnieken besproken om een antwoord te kunnen formuleren op de onderzoeksvragen.

In Hoofdstuk~\ref{ch:proofofconcept} wordt het proof-of-concept uitgewerkt en getest in verschillende scenario’s om de effectiviteit van de voorgestelde oplossing te valideren.

In Hoofdstuk~\ref{ch:evaluatie} worden de resultaten van de simulaties en prestatiemetingen geanalyseerd en vergeleken met de theoretische verwachtingen.

In Hoofdstuk~\ref{ch:conclusie} worden de belangrijkste bevindingen samengevat, een antwoord geformuleerd op de onderzoeksvragen en aanbevelingen gegeven voor toekomstig onderzoek binnen dit domein.
\input{standvanzaken}
%%=============================================================================
%% Methodologie
%%=============================================================================

\chapter{\IfLanguageName{dutch}{Methodologie}{Methodology}}%
\label{ch:methodologie}

%% TODO: In dit hoofstuk geef je een korte toelichting over hoe je te werk bent
%% gegaan. Verdeel je onderzoek in grote fasen, en licht in elke fase toe wat
%% de doelstelling was, welke deliverables daar uit gekomen zijn, en welke
%% onderzoeksmethoden je daarbij toegepast hebt. Verantwoord waarom je
%% op deze manier te werk gegaan bent.
%% 
%% Voorbeelden van zulke fasen zijn: literatuurstudie, opstellen van een
%% requirements-analyse, opstellen long-list (bij vergelijkende studie),
%% selectie van geschikte tools (bij vergelijkende studie, "short-list"),
%% opzetten testopstelling/PoC, uitvoeren testen en verzamelen
%% van resultaten, analyse van resultaten, ...
%%
%% !!!!! LET OP !!!!!
%%
%% Het is uitdrukkelijk NIET de bedoeling dat je het grootste deel van de corpus
%% van je bachelorproef in dit hoofstuk verwerkt! Dit hoofdstuk is eerder een
%% kort overzicht van je plan van aanpak.
%%
%% Maak voor elke fase (behalve het literatuuronderzoek) een NIEUW HOOFDSTUK aan
%% en geef het een gepaste titel.

Deze thesis begon met een grondige requirementsanalyse om de functionele en niet-functionele eisen voor de synchronisatie- en back-upomgeving van Electronic Flight Bags (EFB's) in kaart te brengen. Vanuit deze analyse werd een combinatie van literatuurstudie, simulaties en een proof-of-concept (PoC) toegepast om de centrale onderzoeksvraag te beantwoorden:

\textit{Hoe kan een veilige en gecontroleerde omgeving worden ontworpen voor de synchronisatie en back-up van data in Electronic Flight Bags zonder of met beperkte internetverbinding, om gegevensverlies te minimaliseren?}

Deze aanpak garandeerde dat de voorgestelde oplossing niet alleen technisch haalbaar was, maar ook voldeed aan de specifieke eisen en beperkingen van de werkomgeving waarin deze werd toegepast. Hierbij werd rekening gehouden met realistische scenario’s zoals beperkte of onderbroken netwerkverbindingen, beveiligingseisen en operationele continuïteit.

\section{Fase 1: Requirementsanalyse}
De eerste fase van het onderzoek richtte zich op het uitvoeren van een gedetailleerde requirementsanalyse. Dit omvatte de identificatie van de technische en operationele vereisten die nodig waren voor de ontwikkeling van een veilige en flexibele synchronisatie- en back-upomgeving voor EFB's. Essentiële parameters zoals Recovery Time Objective (RTO), Recovery Point Objective (RPO), databeveiliging en operationele continuïteit stonden centraal.

De requirementsanalyse focuste op:
\begin{itemize}
    \item De functionele eisen voor gegevenssynchronisatie en back-upstrategieën.
    \item De niet-functionele vereisten, zoals beveiliging, prestatie-eisen en betrouwbaarheid.
    \item De impact van netwerkbeperkingen en offline operaties.
    \item Evaluatie van de geschiktheid van bestaande oplossingen en technieken, zoals beschreven in de literatuurstudie.
    \item Behoefteanalyse voor compliance met militaire en luchtvaartveiligheidsnormen zoals beschreven in \textcite{ACISAPGSECVEIL001} en \textcite{FAA_AC91-78A}.
\end{itemize}

\textbf{Resultaat:} Een requirementsdocument waarin de technische en operationele eisen van het proof-of-concept werden vastgelegd in \autoref{ch:requirements}.

\section{Fase 2: Literatuurstudie}
De tweede fase omvatte een uitgebreide literatuurstudie waarin best practices en bestaande methodologieën voor gegevenssynchronisatie en back-up in hybride en offline omgevingen werden geanalyseerd. Hierbij werd gebruikgemaakt van bronnen zoals \textcite{Yanamala2024} en \textcite{VinayakBhuvi} die hybride synchronisatie-oplossingen en cloud-integratie beschreven. Er werd extra aandacht besteed aan de implementatie van veilige gegevensoverdracht in operationele settings met beperkte connectiviteit, zoals beschreven door \textcite{SkybraryEFB}.

De literatuurstudie richtte zich op:
\begin{itemize}
    \item De risico’s en beperkingen van synchronisatie en data-overdracht in kwetsbare operationele settings.
    \item Beveiligingsprotocollen en encryptiemethoden zoals aanbevolen door \textcite{NISTFIPS140}.
    \item Best practices voor disaster recovery en back-upbeheer zoals beschreven in \textcite{VeeamRTO} en \textcite{MicrosoftBackup}.
\end{itemize}

\textbf{Resultaat:} Een rapport dat een overzicht bood van relevante technologieën, strategieën en best practices, met een samenvatting van de belangrijkste bevindingen uit de literatuur in \autoref{ch:ch:stand-van-zaken}.

\section{Fase 3: Proof-of-Concept (PoC)}
Op basis van de inzichten uit de eerste twee fasen werd een proof-of-concept ontwikkeld waarin een veilige en gecontroleerde synchronisatie- en back-upomgeving voor EFB’s werd getest. Dit PoC werd uitgevoerd in een gesimuleerde omgeving waarin realistische operationele omstandigheden werden nagebootst.

De testopstelling bestond uit:
\begin{itemize}
    \item Een on-premise serveromgeving gehost in een virtuele infrastructuur via \textbf{VirtualBox}.
    \item Een \textbf{Veeam Backup}-oplossing geconfigureerd voor het maken van back-ups met meetbare RTO en RPO.
    \item Simulatie van netwerkstoringen en het effect hiervan op de synchronisatie.
    \item Encryptie van data-overdracht volgens de \textbf{NIST 800-88} en \textbf{FIPS 140-2} standaarden.
    \item Implementatie van redundante opslagstrategieën in lijn met \textcite{AWSBackup} en \textcite{MicrosoftBackup}.
\end{itemize}

Het proof-of-concept werd getest in de volgende scenario’s:
\begin{itemize}
    \item Dataverlies en herstel via lokale back-ups.
    \item Gegevenssynchronisatie zonder actieve netwerkverbinding.
    \item Beveiligingsincidenten zoals ongeautoriseerde toegangspogingen.
    \item Vergelijking van verschillende back-upstrategieën (lokaal vs. cloud-gebaseerd).
\end{itemize}

\textbf{Resultaat:} Een volledig functionerend proof-of-concept dat realistische scenario’s simuleerde en operationele parameters zoals RTO en RPO mat.

\section{Fase 4: Validatie en Evaluatie}
De vierde fase richtte zich op de validatie en evaluatie van het proof-of-concept. De prestaties werden beoordeeld aan de hand van prestatiemetingen en vergelijking met theoretische modellen uit de literatuurstudie.

De evaluatie omvatte:
\begin{itemize}
    \item Prestatiemetingen van de back-up- en synchronisatiestrategieën.
    \item Analyse van RTO en RPO-prestaties onder verschillende testomstandigheden.
    \item Controle van de naleving van beveiligingsrichtlijnen volgens \textcite{ACISAPGSECVEIL001} en \textcite{AD070001}.
    \item Identificatie van optimalisatiepunten en aanbevelingen voor toekomstige implementaties.
\end{itemize}

\textbf{Resultaat:} Een evaluatierapport waarin de prestaties van het PoC werden geanalyseerd en vergeleken met de theoretische verwachtingen, inclusief aanbevelingen voor toekomstige verbeteringen.

\section{Tools en Technologieën}
Voor de uitvoering van deze bachelorproef werden de volgende tools en technologieën ingezet:
\begin{itemize}
    \item \textbf{VirtualBox}: Voor het creëren van een virtuele on-premise server.
    \item \textbf{Veeam Backup}: Voor het beheren en uitvoeren van back-upstrategieën.
    \item \textbf{Microsoft Azure Backup / AWS Backup}: Voor cloudgebaseerde opslag en redundantie.
    \item \textbf{Python / Excel}: Voor data-analyse en visualisatie van prestatiemetingen.
\end{itemize}




% Voeg hier je eigen hoofdstukken toe die de ``corpus'' van je bachelorproef
% vormen. De structuur en titels hangen af van je eigen onderzoek. Je kan bv.
% elke fase in je onderzoek in een apart hoofdstuk bespreken.

%\input{...}
%\input{...}
%...
%%=============================================================================
%% Proof of Concept
%%=============================================================================

\chapter{\IfLanguageName{dutch}{Proof-of-Concept}{Proof-of-Concept}}%
\label{ch:proofofconcept}

In dit hoofdstuk wordt het proof-of-concept uitgewerkt en getest in verschillende scenario’s om de effectiviteit van de voorgestelde oplossing te valideren.
%%=============================================================================
%% Evaluatie
%%=============================================================================

\chapter{\IfLanguageName{dutch}{Evaluatie}{Evaluation}}%
\label{ch:evaluatie}

In dit hoofdstuk worden de resultaten van de simulaties en prestatiemetingen geanalyseerd en vergeleken met de theoretische verwachtingen.
\input{conclusie}

%---------- Bijlagen -----------------------------------------------------------

\appendix

\chapter{Onderzoeksvoorstel}

Het onderwerp van deze bachelorproef is gebaseerd op een onderzoeksvoorstel dat vooraf werd beoordeeld door de promotor. Dat voorstel is opgenomen in deze bijlage.

%% TODO: 
%\section*{Samenvatting}

% Kopieer en plak hier de samenvatting (abstract) van je onderzoeksvoorstel.

Dit onderzoek richt zich op het ontwerpen van een veilige en gecontroleerde omgeving voor de synchronisatie en back-up van data in Electronic Flight Bags (EFB's). Deze tablets, essentieel voor piloten, spelen een cruciale rol in het moderniseren van operationele processen, zoals het raadplegen van vluchtplannen, kaarten en andere essentiële informatie. De context waarin deze apparaten worden gebruikt, varieert sterk: van volledig internettoegang tot situaties waarin connectiviteit beperkt is of volledig ontbreekt, zoals in afgelegen gebieden of tijdens buitenlandse operaties. De uitdaging hierbij is het garanderen van gegevensintegriteit, continuïteit en veiligheid, zonder risico op datalekken naar onbevoegde partijen.

De centrale onderzoeksvraag luidt: Hoe kan een veilige en gecontroleerde omgeving worden ontworpen voor de synchronisatie en back-up van data in Electronic Flight Bags zonder of met beperkte internetverbinding, om gegevensverlies te minimaliseren?

De doelstelling van dit onderzoek is om een proof-of-concept (PoC) te ontwikkelen die een gesimuleerde operationele omgeving nabootst. Deze omgeving zal worden getest op kernparameters zoals Recovery Time Objective (RTO), Recovery Point Objective (RPO), en beveiligingsniveaus. Denk hierbij aan scenario's waarbij data moet worden gesynchroniseerd via een lokaal netwerk of zelfs handmatig via een veilige opslagdrager, zoals een USB-stick of een secure enclave op het apparaat. Tools zoals Veeam Backup en vergelijkbare oplossingen worden onderzocht en ingezet om de back-upprocessen te optimaliseren.

Dit onderzoek maakt gebruik van een combinatie van literatuurstudie en praktische experimenten. De literatuurstudie richt zich op de huidige best practices voor gegevensbeheer in hybride en offline omgevingen, terwijl de experimenten dienen om de haalbaarheid van de voorgestelde strategie te toetsen in realistische scenario's. Verwachte resultaten omvatten een reeks best practices en richtlijnen die specifiek zijn ontworpen voor operationele omgevingen waarin flexibiliteit en veiligheid cruciaal zijn.

De meerwaarde van dit onderzoek is tweeledig. Enerzijds biedt het praktische inzichten voor IT-beheerders in luchtvaartorganisaties, waar de noodzaak om gevoelige gegevens te beveiligen tegen externe dreigingen groot is. Anderzijds draagt het bij aan het verhogen van de operationele efficiëntie en betrouwbaarheid van EFB-systemen, zodat deze ook in complexe en uitdagende omgevingen probleemloos blijven functioneren.

Met deze aanpak beoogt het onderzoek niet alleen een oplossing voor het synchroniseren en back-uppen van data in operationele contexten, maar ook een blauwdruk voor de implementatie van veilige IT-systemen in de toekomst. Dit maakt het relevant voor organisaties die zich bezighouden met kritieke infrastructuren en waar continuïteit en veiligheid absolute prioriteiten zijn.

% Verwijzing naar het bestand met de inhoud van het onderzoeksvoorstel

%---------- Inleiding ---------------------------------------------------------


% vorig jaar hebt ingediend? Heb je daarbij eventueel samengewerkt met een
% andere student?
% Zo ja, haal dan de tekst hieronder uit commentaar en pas aan.

%\paragraph{Opmerking}

% Dit voorstel is gebaseerd op het onderzoeksvoorstel dat werd geschreven in het
% kader van het vak Research Methods dat ik (vorig/dit) academiejaar heb
% uitgewerkt (met medesturent VOORNAAM NAAM als mede-auteur).
% 

\section{Inleiding}%
\label{sec:inleiding}

%Waarover zal je bachelorproef gaan? Introduceer het thema en zorg dat volgende zaken zeker duidelijk aanwezig zijn:

%\begin{itemize}
  %\item kaderen thema
  %\item de doelgroep
  %\item de probleemstelling en (centrale) onderzoeksvraag
  %\item de onderzoeksdoelstelling
%\end{itemize}

%Denk er aan: een typische bachelorproef is \textit{toegepast onderzoek}, wat betekent dat je start vanuit een concrete probleemsituatie in bedrijfscontext, een \textbf{casus}. Het is belangrijk om je onderwerp goed af te bakenen: je gaat voor die \textit{ene specifieke probleemsituatie} op zoek naar een goede oplossing, op basis van de huidige kennis in het vakgebied.

%De doelgroep moet ook concreet en duidelijk zijn, dus geen algemene of vaag gedefinieerde groepen zoals \emph{bedrijven}, \emph{developers}, \emph{Vlamingen}, enz. Je richt je in elk geval op it-professionals, een bachelorproef is geen populariserende tekst. Eén specifiek bedrijf (die te maken hebben met een concrete probleemsituatie) is dus beter dan \emph{bedrijven} in het algemeen.

%Formuleer duidelijk de onderzoeksvraag! De begeleiders lezen nog steeds te veel voorstellen waarin we geen onderzoeksvraag terugvinden.

%Schrijf ook iets over de doelstelling. Wat zie je als het concrete eindresultaat van je onderzoek, naast de uitgeschreven scriptie? Is het een proof-of-concept, een rapport met aanbevelingen, \ldots Met welk eindresultaat kan je je bachelorproef als een succes beschouwen?


%---------- Stand van zaken ---------------------------------------------------

In de context van de toenemende adoptie van cloud computing staan veel bedrijven voor de uitdaging om hun bestaande on-premise infrastructuren effectief te integreren met cloudoplossingen. Dit onderzoek richt zich op het ontwerpen van een hybride cloud-backupstrategie die zowel robuuste gegevensbeveiliging als bedrijfscontinuïteit waarborgt. De doelgroep van deze bachelorproef bestaat uit IT-professionals en systeembeheerders die verantwoordelijk zijn voor het beheer van hybride IT-infrastructuren binnen middelgrote tot grote organisaties. Deze bedrijven opereren in complexe omgevingen waarin zowel fysieke als cloudgebaseerde systemen moeten worden beschermd tegen dataverlies en onverwachte uitval, veroorzaakt door factoren zoals hardwarefalen, cyberaanvallen en natuurrampen.

De probleemstelling die centraal staat in dit onderzoek is de vraag hoe bedrijven in een hybride cloudomgeving een effectieve backupstrategie kunnen implementeren die zowel voldoet aan de moderne beveiligingsvereisten als aan de operationele eisen van minimale downtime en snelle hersteltijden. Traditionele back-upmethoden schieten vaak tekort in hybride omgevingen door de complexiteit van het beheer van meerdere locaties en de noodzaak om data consistent te synchroniseren tussen on-premise systemen en cloudopslag. Deze uitdagingen worden verder benadrukt door \textcite{Yanamala2024}, die de noodzaak voor een geïntegreerde aanpak voor hybride cloudstrategieën onderstreept. De centrale onderzoeksvraag luidt dan ook: \textit{Hoe kan een hybride cloud-backupstrategie worden ontworpen om zowel de beveiliging van gegevens als de bedrijfscontinuïteit te optimaliseren?}

De doelstelling van dit onderzoek is om een concrete en schaalbare back-upoplossing te ontwikkelen die specifiek gericht is op hybride cloudomgevingen. Het eindresultaat zal bestaan uit een proof-of-concept waarbij verschillende technologieën en tools worden getest en geëvalueerd op hun effectiviteit met betrekking tot \textit{Recovery Point Objectives} (RPO), \textit{Recovery Time Objectives} (RTO), en beveiligingseisen. Dit sluit aan bij de bevindingen van \textcite{VinayakBhuvi}, die de voordelen van cloudgebaseerde disaster recovery benadrukken. Deze bachelorproef zal, naast het proof-of-concept, ook resulteren in een rapport met aanbevelingen en best practices die IT-professionals kunnen helpen bij het verbeteren van hun back-up- en disaster recovery-plannen in hybride omgevingen.

Het onderzoek zal worden gebaseerd op een concrete casus uit het werkveld, waarin een middelgrote organisatie worstelt met het implementeren van een hybride back-upoplossing. Door de huidige technologieën te evalueren en te testen, zal dit onderzoek bijdragen aan de bestaande kennis en concrete richtlijnen bieden voor het effectief inzetten van hybride cloudoplossingen, zoals besproken door \textcite{ARASET48PP100_116}.
​
\section{Literatuurstudie}% 
\label{sec}

Hybride cloudomgevingen winnen aan populariteit onder organisaties die zowel de voordelen van on-premise infrastructuren als die van cloudoplossingen willen benutten. Deze combinatie biedt aanzienlijke voordelen, zoals verbeterde flexibiliteit, schaalbaarheid en de mogelijkheid om gegevens veilig op verschillende locaties op te slaan. Volgens \textcite{Yanamala2024} stelt het gebruik van hybride cloudstrategieën organisaties in staat om data op meerdere plekken te beheren, wat cruciaal is voor het verminderen van de kans op dataverlies en het waarborgen van de operationele continuïteit. Deze aanpak stelt bedrijven niet alleen in staat om snel te herstellen van storingen, maar verhoogt ook hun weerbaarheid tegen potentiële rampen.

Echter, de implementatie van een hybride cloud-backupstrategie gaat niet zonder uitdagingen. In het onderzoek van \textcite{ARASET48PP100_116} wordt benadrukt dat organisaties een grondige analyse van hun huidige IT-infrastructuur moeten uitvoeren voordat ze overstappen op hybride oplossingen. Het is noodzakelijk om de specifieke vereisten van de bedrijfsprocessen goed in kaart te brengen, zodat de gekozen strategie aansluit bij de operationele behoeften. Daarnaast is het regelmatig testen van disaster recovery-plannen van vitaal belang om te verzekeren dat deze strategieën effectief zijn en in lijn blijven met de bedrijfsdoelstellingen.

Een belangrijk voordeel van hybride cloudoplossingen is de elasticiteit van de cloudinfrastructuur, die bedrijven in staat stelt om hun capaciteiten aan te passen op basis van de behoeften. Volgens \textcite{VinayakBhuvi} kunnen organisaties hun infrastructuur eenvoudig opschalen wanneer dat nodig is, zonder de kosten en complexiteit die gepaard gaan met het onderhoud van een volledige duplicaat van hun productieomgeving. Dit biedt niet alleen een kostenvoordeel, maar stelt bedrijven ook in staat om hun resources efficiënter te gebruiken, wat leidt tot verbeterde prestaties en lagere operationele kosten.

Desondanks brengt de integratie van cloudtechnologieën in bestaande systemen ook aanzienlijke uitdagingen met zich mee, waaronder compatibiliteitsproblemen en verhoogde beveiligingsrisico's. Het is essentieel dat organisaties deze risico's proactief beheren door middel van het opstellen van duidelijke beleidslijnen en het ontwikkelen van robuuste strategieën voor gegevensbeveiliging en -herstel. Het onderzoek van \textcite{Yanamala2024} wijst erop dat een gedegen aanpak van deze risico's niet alleen de kans op incidenten verkleint, maar ook bijdraagt aan het vertrouwen van stakeholders in de gekozen cloudstrategieën.

In de context van deze literatuurstudie is het evident dat hybride cloud-backupstrategieën aanzienlijke voordelen bieden, mits ze zorgvuldig worden geïmplementeerd. Het succes van deze strategieën hangt af van een grondige evaluatie van de bestaande infrastructuur en een strategische aanpak die rekening houdt met de dynamische aard van de IT-omgeving. De inzichten verkregen uit de huidige literatuur vormen een waardevolle basis voor het ontwikkelen van effectieve hybride cloudoplossingen die de continuïteit van bedrijfsprocessen waarborgen en de weerbaarheid van organisaties tegen verstoringen vergroten.


%---------- Methodologie ------------------------------------------------------
\section{Methodologie}%
\label{sec:methodologie}

%Hier beschrijf je hoe je van plan bent het onderzoek te voeren. Welke onderzoekstechniek ga je toepassen om elk van je onderzoeksvragen te beantwoorden? Gebruik je hiervoor literatuurstudie, interviews met belanghebbenden (bv.~voor requirements-analyse), experimenten, simulaties, vergelijkende studie, risico-analyse, PoC, \ldots?

%Valt je onderwerp onder één van de typische soorten bachelorproeven die besproken zijn in de lessen Research Methods (bv.\ vergelijkende studie of risico-analyse)? Zorg er dan ook voor dat we duidelijk de verschillende stappen terug vinden die we verwachten in dit soort onderzoek!

%Vermijd onderzoekstechnieken die geen objectieve, meetbare resultaten kunnen opleveren. Enquêtes, bijvoorbeeld, zijn voor een bachelorproef informatica meestal \textbf{niet geschikt}. De antwoorden zijn eerder meningen dan feiten en in de praktijk blijkt het ook bijzonder moeilijk om voldoende respondenten te vinden. Studenten die een enquête willen voeren, hebben meestal ook geen goede definitie van de populatie, waardoor ook niet kan aangetoond worden dat eventuele resultaten representatief zijn.

%Uit dit onderdeel moet duidelijk naar voor komen dat je bachelorproef ook technisch voldoen\-de diepgang zal bevatten. Het zou niet kloppen als een bachelorproef informatica ook door bv.\ een student marketing zou kunnen uitgevoerd worden.

%Je beschrijft ook al welke tools (hardware, software, diensten, \ldots) je denkt hiervoor te gebruiken of te ontwikkelen.

%Probeer ook een tijdschatting te maken. Hoe lang zal je met elke fase van je onderzoek bezig zijn en wat zijn de concrete \emph{deliverables} in elke fase?

Deze thesis maakt gebruik van een combinatie van literatuurstudie, simulaties en een proof-of-concept (PoC) om de onderzoeksvraag te beantwoorden: *Hoe kan een hybride cloud-backupstrategie worden ontworpen die voldoet aan zowel de eisen voor bedrijfscontinuïteit als gegevensbeveiliging?*

\subsection{Fase 1: Literatuurstudie}
De eerste fase van het onderzoek omvat een uitgebreide literatuurstudie, gericht op de huidige technieken en best practices op het gebied van hybride cloud-back-upstrategieën en disaster recovery. In deze fase zal academische en professionele vakliteratuur worden geanalyseerd om de belangrijkste parameters en uitdagingen in dit domein te identificeren. Deze studie zal ongeveer drie weken duren en resulteert in een overzicht van relevante technologieën en strategieën.

\subsection{Fase 2: Simulaties en Proof-of-Concept (PoC)}
In de tweede fase van het onderzoek zal een proof-of-concept worden ontwikkeld waarin een hybride cloud-back-upoplossing wordt gesimuleerd. Hiervoor wordt gebruikgemaakt van Azure Backup en Azure Site Recovery om de cloudcomponent van de oplossing te realiseren. Een virtuele on-premise serveromgeving zal worden opgezet met VirtualBox, waarbij de on-premise infrastructuur wordt nagebootst.

Het proof-of-concept zal worden getest op verschillende scenario's, zoals dataverlies en systeemuitval, om de effectiviteit van de back-upstrategie te evalueren. Tijdens deze experimenten zullen de parameters Recovery Point Objective (RPO) en Recovery Time Objective (RTO) worden gemeten om de prestaties van de back-upoplossing te analyseren. Deze fase is gepland om ongeveer vier weken te duren, met als eindresultaat een werkend prototype dat de robuustheid van de back-upstrategie valideert.

\subsection{Fase 3: Validatie en Evaluatie}
In de derde fase zullen de resultaten van het PoC worden gevalideerd door middel van prestatiemetingen en analyses. De prestaties zullen worden geëvalueerd op basis van de herstelbaarheid en veiligheid van de gegevens, en er zal een vergelijking worden gemaakt met de theoretische verwachtingen uit de literatuurstudie. Python en Excel zullen worden gebruikt voor het analyseren en visualiseren van de prestatiegegevens. Deze fase neemt twee weken in beslag en resulteert in een evaluatierapport dat de effectiviteit van de oplossing samenvat.

\subsection{Tools en Technologieën}
Voor de uitvoering van deze bachelorproef worden de volgende tools en technologieën ingezet:
\begin{itemize}
    \item \textbf{VirtualBox}: voor het creëren van een virtuele on-premise server.
    \item \textbf{Azure Backup / Site Recovery}: voor het beheren en uitvoeren van back-upstrategieën.
    \item \textbf{Cloudopslagdienst}: als cloudoplossing voor het opslaan van back-ups.
    \item \textbf{Python / Excel}: voor data-analyse en visualisatie van de resultaten.
\end{itemize}

\subsection{Tijdsplanning en Deliverables}
Het onderzoek is verdeeld in de volgende fasen, met de bijbehorende tijdsplanning:
\begin{itemize}
    \item \textbf{Week 1-3}: Literatuurstudie – Resultaat: uitgebreid literatuuroverzicht.
    \item \textbf{Week 4-7}: Simulatie en Proof-of-Concept – Resultaat: werkend prototype van de back-upstrategie.
    \item \textbf{Week 8-9}: Validatie en Evaluatie – Resultaat: evaluatierapport met prestatieanalyse.
\end{itemize}


%---------- Verwachte resultaten ----------------------------------------------
\section{Verwacht resultaat, conclusie}%
\label{sec:verwachte_resultaten}

%Hier beschrijf je welke resultaten je verwacht. Als je metingen en simulaties uitvoert, kan je hier al mock-ups maken van de grafieken samen met de verwachte conclusies. Benoem zeker al je assen en de onderdelen van de grafiek die je gaat gebruiken. Dit zorgt ervoor dat je concreet weet welk soort data je moet verzamelen en hoe je die moet meten.

%Wat heeft de doelgroep van je onderzoek aan het resultaat? Op welke manier zorgt jouw bachelorproef voor een meerwaarde?

%Hier beschrijf je wat je verwacht uit je onderzoek, met de motivatie waarom. Het is \textbf{niet} erg indien uit je onderzoek andere resultaten en conclusies vloeien dan dat je hier beschrijft: het is dan juist interessant om te onderzoeken waarom jouw hypothesen niet overeenkomen met de resultaten.

In dit onderzoek verwacht ik dat de implementatie van een hybride cloud-backupstrategie zal resulteren in een oplossing die zowel voldoet aan de eisen voor bedrijfscontinuïteit als gegevensbeveiliging. De belangrijkste parameters die gemeten worden, zijn de \textbf{Recovery Point Objective (RPO)} en de \textbf{Recovery Time Objective (RTO)}. De RPO geeft aan hoe recent de back-updata moet zijn, terwijl de RTO de hersteltijd na een incident meet.

Op basis van de simulaties verwacht ik dat de RPO binnen enkele minuten tot uren kan worden gehouden, afhankelijk van de frequentie van de back-ups. De RTO zal naar verwachting variëren tussen enkele minuten en een paar uur, afhankelijk van de grootte van de data en de snelheid van het herstelproces. Ik zal hierbij verschillende back-upscenario's testen, waaronder volledig, incrementeel en differentieel back-upherstel.

\subsection{Mock-up van Verwachte Resultaten}
Voor de grafieken verwacht ik bijvoorbeeld een lijn- of staafdiagram te maken waarin op de X-as de tijd (in uren) wordt weergegeven en op de Y-as de hersteltijd (RTO) in minuten of uren wordt gemeten. Elke lijn in de grafiek zal verschillende back-upmethoden (bijvoorbeeld volledig versus incrementeel) vertegenwoordigen. Dit biedt een duidelijk visueel overzicht van hoe de hersteltijd varieert per methode.

\subsection{Waarde voor de Doelgroep}
De meerwaarde van dit onderzoek voor IT-professionals en organisaties is dat het een duidelijk beeld geeft van hoe een hybride cloud-backupstrategie kan worden geïmplementeerd met een focus op het verbeteren van bedrijfscontinuïteit en gegevensbeveiliging. Dit onderzoek biedt een praktische gids voor het implementeren van een back-upstrategie die zowel on-premise als cloudopslag combineert, en helpt bedrijven bij het minimaliseren van downtime en gegevensverlies bij storingen of rampen.

\subsection{Conclusie}
Het onderzoek zal inzicht geven in de efficiëntie van hybride cloud-back-upsystemen en de afwegingen die moeten worden gemaakt tussen kosten, snelheid en betrouwbaarheid. Zelfs als de resultaten afwijken van de verwachtingen, is dit waardevol omdat het aantoont welke factoren mogelijk nog geoptimaliseerd moeten worden en waarom bepaalde strategieën minder goed werken in specifieke omgevingen. Deze analyse biedt dus belangrijke richtlijnen voor bedrijven die soortgelijke back-upsystemen willen implementeren.




%%---------- Andere bijlagen --------------------------------------------------
% TODO: Voeg hier eventuele andere bijlagen toe. Bv. als je deze BP voor de
% tweede keer indient, een overzicht van de verbeteringen t.o.v. het origineel.
%\chapter{Rapport Literatuurstudie}
%\label{ch:rapport-literatuurstudie}
%\documentclass{hogent-article}
\usepackage[dutch]{babel}
\usepackage{hyperref}
\usepackage{csquotes}
\usepackage[backend=biber,style=authoryear]{biblatex}
\addbibresource{bachproef.bib}

\title{Uitgebreide Literatuurstudie: Veilige Synchronisatie- en Back-upstrategieën voor Electronic Flight Bags}
\author{Stein Van Driessche}
\email{stein.vandriesschen@student.hogent.be}
\date{\today}
\supervisor[Co-promotor]{T. De Quick (Quick IT GCV, \href{mailto:toon\_dequick@hotmail.com}{toon\_dequick@hotmail.com})
    
    \begin{document}
        
        \maketitle
        
        \section{Inleiding}
        
        Electronic Flight Bags (EFB’s) zijn een essentieel onderdeel geworden van moderne luchtvaartoperaties, waarbij ze traditionele papieren handleidingen vervangen door digitale systemen die piloten voorzien van vluchtinformatie, navigatiegegevens en operationele richtlijnen. Hoewel deze technologieën tal van voordelen bieden, brengen ze ook aanzienlijke uitdagingen met zich mee, vooral op het gebied van **data-integriteit, back-upstrategieën en synchronisatiemethoden** in omgevingen met beperkte internetconnectiviteit.
        
        Het ontbreken van een continue netwerkverbinding in militaire en commerciële luchtvaartomgevingen vereist een robuuste strategie voor **gegevensbeheer en herstel**. Dit onderzoek baseert zich op uitgebreide literatuurstudies om inzicht te krijgen in best practices en innovatieve technologieën op het gebied van **hybride back-upsystemen, versleutelde synchronisatie en veerkrachtige dataopslagmodellen**.
        
        \section{Dataopslag en Synchronisatiestrategieën voor EFB’s}
        
        Een van de grootste uitdagingen bij het gebruik van EFB’s is het waarborgen van **gegevensbeschikbaarheid en beveiliging** in situaties waarin netwerkverbindingen onbetrouwbaar of afwezig zijn. Volgens \textcite{Yanamala2024} kunnen hybride opslagsystemen, waarbij gegevens zowel **lokaal als in de cloud** worden opgeslagen, een oplossing bieden. Dit model maakt het mogelijk om gegevens lokaal toegankelijk te houden en periodiek te synchroniseren met een gecentraliseerde opslagomgeving zodra een verbinding beschikbaar is.
        
        \subsection{Hybride Cloud- en Lokale Opslagmodellen}
        
        Volgens \textcite{AWSBackup} bieden multi-tier opslagmodellen een extra laag redundantie door kritieke data zowel lokaal als in de cloud op te slaan. Dit zorgt ervoor dat in geval van systeemuitval een back-up snel kan worden hersteld zonder afhankelijkheid van een internetverbinding. \textcite{MicrosoftBackup} benadrukt dat een **hybride aanpak**, waarbij **on-premise opslag en cloudoplossingen** worden gecombineerd, de meest efficiënte manier is om data-integriteit te garanderen.
        
        Een effectieve oplossing is het gebruik van **incremental back-ups**, waarbij alleen gewijzigde gegevens worden gesynchroniseerd in plaats van volledige datasets. Dit vermindert de belasting op het netwerk en verkort de synchronisatietijd aanzienlijk \autocite{VeeamRTO}.
        
        \subsection{Zero-Trust Architectuur en Gegevensbeveiliging}
        
        Volgens \textcite{NISTFIPS140} is **gegevensversleuteling** een cruciale vereiste voor het veilig opslaan en synchroniseren van EFB-data. Moderne encryptiemethoden zoals **AES-256 en TLS** zorgen ervoor dat alleen geautoriseerde gebruikers toegang krijgen tot de gegevens.
        
        Een **zero-trust beveiligingsmodel** kan volgens \textcite{VinayakBhuvi} helpen om ongeautoriseerde toegang tot kritieke data te voorkomen. Dit model vereist een strikte authenticatie voor **elke interactie** met het systeem en voorkomt datalekken door alleen gegevensoverdracht toe te staan via **gecontroleerde, geverifieerde kanalen**.
        
        \section{Recovery Time Objective (RTO) en Recovery Point Objective (RPO)}
        
        Het optimaliseren van **Recovery Time Objective (RTO)** en **Recovery Point Objective (RPO)** is essentieel om operationele continuïteit te waarborgen. RTO geeft de maximale tijdsduur aan die een systeem kan uitvallen zonder operationele verstoring, terwijl RPO bepaalt hoeveel data maximaal verloren mag gaan in geval van een crash.
        
        \textcite{VeeamRTO} benadrukt dat het **automatiseren van back-ups** en het implementeren van een **failover-systeem** de efficiëntie van gegevensherstel kan verbeteren. \textcite{MicrosoftBackup} stelt dat **snapshot-gebaseerde back-ups** een snellere hersteltijd mogelijk maken dan traditionele full-backups.
        
        \section{Regelgeving en Normen voor EFB-gebruik}
        
        De **FAA-richtlijnen** \autocite{FAA_AC91-78A} specificeren dat alle EFB-systemen moeten voldoen aan **strikte beveiligingsnormen**, waaronder versleutelde opslag en toegangscontroles. Dit sluit aan bij de **NAVO-beveiligingsnormen** \autocite{AD070001}, die aanvullende maatregelen voorschrijven voor het beschermen van **geclassificeerde luchtvaartdata**.
        
        Volgens de **Belgische militaire richtlijnen** \autocite{ACISAPGSECVEIL001} moet geclassificeerde data worden opgeslagen met **versleuteling op het hoogste beveiligingsniveau**, terwijl **incidentresponsprocedures** aanwezig moeten zijn om **cyberdreigingen** direct aan te pakken.
        
        \section{Beschikbare Technologieën en Tools voor Synchronisatie en Back-up}
        
        Een overzicht van beschikbare softwareoplossingen voor het beheer van EFB-back-ups en synchronisatie:
        
        \begin{itemize}
            \item **Veeam Backup**: Ondersteunt **hybride cloudback-ups** en biedt een snelle herstelprocedure \autocite{VeeamRTO}.
            \item **Azure Backup**: Bevat ingebouwde compliance-functionaliteiten en versleutelde opslag \autocite{MicrosoftBackup}.
            \item **AWS Backup**: Multi-tier back-upstrategie met focus op **schaalbaarheid en beveiliging** \autocite{AWSBackup}.
            \item **Logipad EFB**: Specifiek ontworpen voor **offline synchronisatie in luchtvaartomgevingen** zonder internettoegang \autocite{LogipadEFB}.
        \end{itemize}
        
        \section{Toekomstige Ontwikkelingen en Best Practices}
        
        Op basis van de literatuur kunnen **verschillende best practices** worden geïdentificeerd:
        
        \begin{itemize}
            \item **Gebruik van blockchain-technologie** voor onwijzigbare logs en **tamper-proof back-ups** \autocite{VinayakBhuvi}.
            \item **AI-gebaseerde anomaly detection** om afwijkingen in gegevensoverdracht te detecteren en te mitigeren.
            \item **Implementatie van autonome back-upsystemen** die functioneren zonder menselijke tussenkomst en zelfherstellende capaciteiten bevatten.
        \end{itemize}
        
        Volgens \textcite{Abdelaziz48PP100_116} kunnen **automatische validatiemechanismen** helpen om gegevensintegriteit te bewaken, zelfs in offline omgevingen.
        
        \section{Conclusie}
        
        Deze literatuurstudie toont aan dat een **hybride back-upstrategie**, gecombineerd met **zero-trust beveiliging en versleutelde gegevensopslag**, de meest effectieve oplossing biedt voor het beschermen van EFB-data. Tegelijkertijd moeten deze oplossingen voldoen aan bestaande regelgeving, zoals de **FAA- en NAVO-veiligheidsrichtlijnen**, om de betrouwbaarheid en veiligheid van luchtvaartgegevens te garanderen.
        
        Door **nieuwe technologieën**, zoals **blockchain** en **AI-gebaseerde monitoring**, te integreren in toekomstige EFB-systemen, kan de veerkracht van luchtvaartdata aanzienlijk worden verhoogd. Deze inzichten vormen een solide basis voor verdere ontwikkeling van **betrouwbare, efficiënte en veilige synchronisatie- en back-upstrategieën** in luchtvaartomgevingen.
        
        \printbibliography
        
    \end{document}

\chapter{Requirements}
\label{ch:requirements}
%==============================================================================
% Requirements Document for Bachelor Thesis
%=======================================================================
    \section{Inleiding}
    Dit requirementsdocument richt zich op de technische en functionele vereisten voor het ontwikkelen van een veilige en flexibele synchronisatie- en back-upstrategie voor Electronic Flight Bags (EFB's). De nadruk ligt op het functioneren in omgevingen met slechte of geen internetverbinding, terwijl wordt gegarandeerd dat synchronisatie en back-ups correct en volledig worden uitgevoerd.
    
    Het doel van dit document is om de continuïteit en integriteit van de EFB's te waarborgen, aangezien deze tablets essentieel zijn voor de operationele werking van het vliegtuig. Wanneer een EFB beschadigd raakt of niet operationeel is, kan dit directe gevolgen hebben voor de luchtwaardigheid van het vliegtuig.
    
    Daarnaast dient dit document als een basis voor het ontwerpen en ontwikkelen van het proof-of-concept (PoC). De vereisten die hierin worden beschreven, zorgen ervoor dat het PoC voldoet aan de operationele en technische eisen en een oplossing biedt voor de uitdagingen van synchronisatie en back-up in operationele contexten. Het probleemkader wordt kort aangestipt, maar verdere details zijn opgenomen in de hoofdtekst van de thesis.
    
    \section{Doelstellingen}
    De hoofddoelstelling van dit onderzoek is het ontwikkelen van een veilige en gecontroleerde omgeving die:
    \begin{itemize}
        \item Operationeel blijft in kwetsbare en uitdagende contexten, zoals afgelegen locaties en onstabiele netwerkomgevingen.
        \item Hersteltijden (\textit{Recovery Time Objective, RTO}) en dataverlies (\textit{Recovery Point Objective, RPO}) minimaliseert.
        \item Het ontwerpen en testen van een proof-of-concept (PoC) dat de voorgestelde strategieën valideert door middel van simulaties en prestatie-evaluaties in realistische scenario’s.
        \item Het evalueren van welke technologieën en procedures kunnen bijdragen aan een veilige en efficiënte gegevensoverdracht, inclusief versleuteling en failover-mechanismen, in situaties met beperkte connectiviteit.
    \end{itemize}
    
    \section{Functionele Vereisten}
    
    \subsection{Synchronisatie}
    \begin{itemize}
        \item De oplossing moet data kunnen synchroniseren via een lokaal netwerk, Cloudservice of veilige overdracht via USB.
        \item Er moet ondersteuning zijn voor zowel handmatige als automatische synchronisatieprocessen.
        \item Encryptie moet worden toegepast tijdens dataoverdracht.
        \item De oplossing moet compatibel zijn met verschillende netwerkomstandigheden, zoals publieke wifi, satellietverbindingen, of volledig offline scenario's.
        \item Er moet een systeem zijn om conflicten te beheren bij het synchroniseren van gegevens, zoals dubbele bestanden of gewijzigde versies.
        \item Na elke synchronisatie moet de integriteit van de gegevens worden gecontroleerd om te garanderen dat alle bestanden correct en volledig zijn overgedragen.
    \end{itemize}
    
    \subsection{Back-ups}
    \begin{itemize}
        \item Dagelijkse back-ups van operationele gegevens moeten worden gegarandeerd.
        \item Back-ups moeten zowel lokaal als in een gecentraliseerde omgeving kunnen worden opgeslagen.
        \item Er moet een logging- en monitoringsysteem zijn om de status van back-ups te verifiëren.
        \item Het moet mogelijk zijn om verschillende back-upfrequenties in te stellen (bijvoorbeeld per uur, dagelijks, wekelijks).
        \item De oplossing moet meerdere versies van dezelfde bestanden kunnen bewaren, zodat herstel mogelijk is naar een eerdere staat.
        \item Periodieke validatie van de back-ups moet worden uitgevoerd om te garanderen dat de opgeslagen gegevens bruikbaar zijn.
        \item Back-ups moeten eenvoudig en snel kunnen worden hersteld naar de oorspronkelijke of een nieuwe EFB.
        \item Als een back-up mislukt, moet de oplossing automatisch een alternatieve back-uplocatie kiezen.
    \end{itemize}
    
    \subsection{Algemene Functionele Vereisten}
    \begin{itemize}
        \item De interface moet intuïtief en eenvoudig te gebruiken zijn voor eindgebruikers, zonder complexe configuraties.
        \item Gedetailleerde logboeken moeten worden bijgehouden voor alle synchronisatie- en back-upprocessen, inclusief fouten en waarschuwingen.
        \item De oplossing moet compatibel zijn met bestaande systemen, zoals Veeam Backup of vergelijkbare technologieën.
        \item De oplossing moet voldoen aan relevante standaarden en regelgeving binnen de luchtvaartsector.
    \end{itemize}

    
    \section{Technische Vereisten}
    \begin{itemize}
    \item De oplossing moet volledig compatibel zijn met bestaande infrastructuren, waaronder back-upsoftware zoals Veeam Backup en andere gangbare tools, om integratie te vergemakkelijken.
    \item Parameters zoals Recovery Time Objective (RTO) en Recovery Point Objective (RPO) moeten meetbaar zijn, met mogelijkheden om deze parameters aan te passen op basis van specifieke operationele eisen en scenario's.
    \item De implementatie moet robuust functioneren bij beperkte bandbreedte en onbetrouwbare netwerkverbindingen, en mag niet afhankelijk zijn van een constante verbinding.
    \item Gegevens moeten worden opgeslagen in een formaat dat voldoet aan standaardreglementen voor beveiliging, zoals encryptie van gevoelige informatie en naleving van relevante regelgeving.
    \item Het systeem moet ondersteuning bieden voor veilige gegevensoverdracht, inclusief versleuteling tijdens het transport en controlemechanismen om gegevensintegriteit te waarborgen.
    \item Er moet een mogelijkheid zijn om data lokaal op te slaan in situaties waarin externe opslag niet beschikbaar of onbetrouwbaar is.
    \item Het ontwerp moet schaalbaar zijn, zodat toekomstige groei in datavolume en complexe operationele vereisten ondersteund kunnen worden zonder herziening van de basisarchitectuur.
    \item Het systeem moet eenvoudig te monitoren en te beheren zijn, met ingebouwde functies voor foutmeldingen, logboeken en prestatiestatistieken.
    \item Authenticatie- en autorisatiemechanismen moeten worden geïntegreerd om toegang tot de oplossing te beperken tot bevoegde gebruikers en apparaten.
    \item Het systeem moet failover-mogelijkheden bevatten, zodat herstel van synchronisatie of back-ups mogelijk blijft bij onverwachte systeemstoringen.
    \end{itemize}

    
    \section{Beperkingen}
    \begin{itemize}
    \item De oplossing mag geen afhankelijkheid hebben van constante internettoegang.
    \item Gegevensoverdracht en -synchronisatie moeten robuust blijven in onstabiele omgevingen.
    \item De implementatie mag geen significante impact hebben op de batterijduur van de EFB's.
    \item De oplossing moet eenvoudig te implementeren zijn zonder complexe aanpassingen aan de bestaande workflows.
    \item Het gebruik van externe opslagmedia, zoals USB-sticks, moet veilig en gecontroleerd verlopen, inclusief encryptie en authenticatie.
    \item Het systeem moet schaalbaar zijn om te voldoen aan de groeiende eisen van dataopslag en synchronisatie.
    \item Er mogen geen kritieke afhankelijkheden zijn van specifieke commerciële oplossingen om vendor lock-in te vermijden.
    \item Het systeem moet voldoen aan relevante regelgeving en interne richtlijnen voor gegevensbeveiliging en luchtvaartstandaarden.
    \item Testen en validatie moeten uitvoerbaar zijn in gesimuleerde operationele omgevingen zonder directe afhankelijkheid van live systemen.
    \end{itemize}


%%---------- Backmatter, referentielijst ---------------------------------------

\backmatter{}

\setlength\bibitemsep{2pt} %% Add Some space between the bibliograpy entries
\printbibliography[heading=bibintoc]

\end{document}
