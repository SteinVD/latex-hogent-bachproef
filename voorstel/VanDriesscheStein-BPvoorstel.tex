%==============================================================================
% Sjabloon onderzoeksvoorstel bachproef
%==============================================================================
% Gebaseerd op document class `hogent-article'
% zie <https://github.com/HoGentTIN/latex-hogent-article>

% Voor een voorstel in het Engels: voeg de documentclass-optie [english] toe.
% Let op: kan enkel na toestemming van de bachelorproefcoördinator!
\documentclass{hogent-article}

% Invoegen bibliografiebestand
\addbibresource{voorstel.bib}

% Informatie over de opleiding, het vak en soort opdracht
\studyprogramme{Professionele bachelor toegepaste informatica}
\course{Bachelorproef}
\assignmenttype{Onderzoeksvoorstel}
% Voor een voorstel in het Engels, haal de volgende 3 regels uit commentaar
% \studyprogramme{Bachelor of applied information technology}
% \course{Bachelor thesis}
% \assignmenttype{Research proposal}

\academicyear{2042-2025} 


\title{Ontwerp van een Hybride Cloud Backupstrategie: Beste Praktijken en Uitdagingen voor Gegevensherstel en Bedrijfscontinuïteit}


\author{Stein Van Driessche}
\email{stein.vandriesschen@student.hogent.be}


% Gaat het om een bachelorproef in samenwerking met een student in een andere
% opleiding? Geef dan de naam en emailadres hier
% \author{Yasmine Alaoui (naam opleiding)}
% \email{yasmine.alaoui@student.hogent.be}


\supervisor[Co-promotor]{T. De Quick (Quick IT GCV, \href{mailto:toon\_dequick@hotmail.com}{toon\_dequick@hotmail.com})}

% Binnen welke specialisatierichting uit 3TI situeert dit onderzoek zich?
% Kies uit deze lijst:
%
% - Mobile \& Enterprise development
% - AI \& Data Engineering
% - Functional \& Business Analysis
% - System \& Network Administrator
% - Mainframe Expert
% - Als het onderzoek niet past binnen een van deze domeinen specifieer je deze
%   zelf
%
\specialisation{System \& Network Administrator}
\keywords{Hybride Cloud, Backupstrategie, Gegevensherstel, Bedrijfscontinuïteit, Disaster Recovery, Recovery Point Objective (RPO), Recovery Time Objective (RTO), Cloudinfrastructuren, On-Premise}


\begin{document}
    
    \begin{abstract}
        Dit onderzoek richt zich op het ontwerpen van een hybride cloud-backupstrategie die optimale bedrijfscontinuïteit en gegevensherstel waarborgt in organisaties die gebruik maken van zowel on-premise als cloud-infrastructuren. De probleemstelling draait om de uitdagingen waarmee organisaties worden geconfronteerd bij het ontwikkelen van een effectieve strategie om gegevens te beveiligen in hybride omgevingen, waarbij downtime en gegevensverlies tot een minimum worden beperkt. De centrale onderzoeksvraag luidt: Hoe kan een hybride cloud-backupstrategie worden ontworpen om een robuuste gegevensherstel- en disaster recovery-oplossing te bieden die voldoet aan moderne prestatie- en beveiligingsvereisten? De doelstelling van het onderzoek is om een concrete, schaalbare en kostenbewuste back-upoplossing te ontwerpen en te valideren via een proof-of-concept (PoC). Dit onderzoek maakt gebruik van een literatuurstudie, aangevuld met technische experimenten waarbij verschillende hybride cloudoplossingen worden getest op parameters zoals Recovery Time Objective (RTO), Recovery Point Objective (RPO), en beveiligingsniveaus. Verwachte resultaten zijn een set van best practices voor het implementeren van een hybride cloud-backupstrategie, evenals een evaluatie van veelgebruikte tools en technieken voor disaster recovery. De meerwaarde van dit onderzoek ligt in het aanbieden van praktische inzichten en richtlijnen die IT-beheerders en bedrijven kunnen gebruiken om hun hybride cloudomgevingen te optimaliseren voor maximale gegevensbeschikbaarheid en beveiliging.
    \end{abstract}
    
    \tableofcontents
    
    % De hoofdtekst van het voorstel zit in een apart bestand, zodat het makkelijk
    % kan opgenomen worden in de bijlagen van de bachelorproef zelf.
    %---------- Inleiding ---------------------------------------------------------


% vorig jaar hebt ingediend? Heb je daarbij eventueel samengewerkt met een
% andere student?
% Zo ja, haal dan de tekst hieronder uit commentaar en pas aan.

%\paragraph{Opmerking}

% Dit voorstel is gebaseerd op het onderzoeksvoorstel dat werd geschreven in het
% kader van het vak Research Methods dat ik (vorig/dit) academiejaar heb
% uitgewerkt (met medesturent VOORNAAM NAAM als mede-auteur).
% 

\section{Inleiding}%
\label{sec:inleiding}

%Waarover zal je bachelorproef gaan? Introduceer het thema en zorg dat volgende zaken zeker duidelijk aanwezig zijn:

%\begin{itemize}
  %\item kaderen thema
  %\item de doelgroep
  %\item de probleemstelling en (centrale) onderzoeksvraag
  %\item de onderzoeksdoelstelling
%\end{itemize}

%Denk er aan: een typische bachelorproef is \textit{toegepast onderzoek}, wat betekent dat je start vanuit een concrete probleemsituatie in bedrijfscontext, een \textbf{casus}. Het is belangrijk om je onderwerp goed af te bakenen: je gaat voor die \textit{ene specifieke probleemsituatie} op zoek naar een goede oplossing, op basis van de huidige kennis in het vakgebied.

%De doelgroep moet ook concreet en duidelijk zijn, dus geen algemene of vaag gedefinieerde groepen zoals \emph{bedrijven}, \emph{developers}, \emph{Vlamingen}, enz. Je richt je in elk geval op it-professionals, een bachelorproef is geen populariserende tekst. Eén specifiek bedrijf (die te maken hebben met een concrete probleemsituatie) is dus beter dan \emph{bedrijven} in het algemeen.

%Formuleer duidelijk de onderzoeksvraag! De begeleiders lezen nog steeds te veel voorstellen waarin we geen onderzoeksvraag terugvinden.

%Schrijf ook iets over de doelstelling. Wat zie je als het concrete eindresultaat van je onderzoek, naast de uitgeschreven scriptie? Is het een proof-of-concept, een rapport met aanbevelingen, \ldots Met welk eindresultaat kan je je bachelorproef als een succes beschouwen?


%---------- Stand van zaken ---------------------------------------------------

In moderne luchtvaartomgevingen spelen Electronic Flight Bags (EFB's) een essentiële rol in het ondersteunen van piloten bij het uitvoeren van vluchten. Deze tablets vervangen papieren handleidingen en vluchtplannen en bieden toegang tot kritieke informatie zoals navigatiekaarten, weersvoorspellingen en operationele richtlijnen. Het gebruik van EFB's brengt echter specifieke uitdagingen met zich mee, vooral in operationele settings waar de beschikbaarheid van internet beperkt of geheel afwezig is, zoals in afgelegen regio's of tijdelijke locaties. Hier wordt data vaak verwerkt onder omstandigheden die betrouwbare synchronisatie en back-ups bemoeilijken.

Een concreet probleem waarmee organisaties in deze context worden geconfronteerd, is hoe zij gevoelige data op EFB's kunnen beveiligen en synchroniseren zonder te vertrouwen op constante internettoegang. Momenteel worden synchronisatieprocessen vaak uitgevoerd via clouddiensten zoals Nextcloud, maar deze oplossingen voldoen niet altijd aan de eisen voor veilige back-ups en operationele betrouwbaarheid. Daarnaast brengt het werken in onveilige netwerkomgevingen, zoals openbare wifi in hotels of lokale computers op geïmproviseerde locaties, extra risico’s met zich mee voor datalekken en gegevensverlies. Het ontbreken van een robuuste en gestandaardiseerde methode vergroot deze kwetsbaarheden en kan de operationele continuïteit in gevaar brengen.

Dit onderzoek richt zich op het ontwikkelen van een oplossing voor deze uitdagingen en wordt gedreven door de centrale onderzoeksvraag:
Hoe kan een veilige en gecontroleerde omgeving worden ontworpen voor de synchronisatie en back-up van data in Electronic Flight Bags zonder of met beperkte internetverbinding, om gegevensverlies te minimaliseren?

Om deze hoofdvraag te beantwoorden, worden diverse deelvragen geformuleerd. Deze richten zich zowel op het identificeren van de problemen als op het ontwikkelen van mogelijke oplossingen.
Welke veiligheidsrisico’s en technische beperkingen zijn inherent aan de huidige methoden voor synchronisatie en back-ups?
Welke technologieën en procedures kunnen bijdragen aan een veilige en efficiënte gegevensoverdracht in afwezigheid van internet?
Door deze vragen te beantwoorden, kan het onderzoek zowel de bestaande uitdagingen analyseren als concrete oplossingen voorstellen.

Het doel van dit onderzoek is om een concrete, schaalbare oplossing te ontwikkelen in de vorm van een proof-of-concept (PoC). Deze oplossing zal verschillende scenario’s simuleren, waaronder data-overdracht via lokale netwerken, het gebruik van veilige bestandsoverdrachten, en de integratie van back-uptechnologieën zoals Veeam. Hierbij worden belangrijke operationele parameters zoals Recovery Time Objectives (RTO) en Recovery Point Objectives (RPO) in kaart gebracht. Deze benadering biedt niet alleen een praktische oplossing voor het probleem, maar draagt ook bij aan de ontwikkeling van richtlijnen en best practices voor het gebruik van EFB's in complexe operationele contexten.

Door de focus te leggen op het ontwerp van een veilige en flexibele omgeving die rekening houdt met wisselende operationele omstandigheden, beoogt dit onderzoek een waardevolle bijdrage te leveren aan het verbeteren van databeheer binnen de luchtvaartsector. De resultaten bieden niet alleen directe praktische inzichten, maar kunnen ook dienen als leidraad voor bredere toepassingen binnen vergelijkbare operationele domeinen.
​
\section{Literatuurstudie}% 
\label{sec}

In operationele omgevingen waar Electronic Flight Bags (EFB's) essentieel zijn, speelt de beschikbaarheid en veiligheid van gegevens een cruciale rol. De integratie van back-up- en synchronisatiestrategieën in dergelijke contexten vereist een aanpak die rekening houdt met de unieke uitdagingen van afwisselende connectiviteit, zoals scenario's waarin geen stabiele internetverbinding beschikbaar is. Volgens \textcite{Yanamala2024} stelt een hybride aanpak organisaties in staat om data lokaal te synchroniseren en op te slaan, terwijl cloudoplossingen de flexibiliteit en schaalbaarheid bieden om deze data te beveiligen en centraal te beheren.

Een belangrijke bevinding uit de literatuur benadrukt dat het waarborgen van gegevensintegriteit in situaties zonder stabiele netwerkverbinding een fundamentele uitdaging vormt. Dit roept de vraag op: Hoe kan de gegevensintegriteit effectief worden gewaarborgd in dergelijke omstandigheden?
Zoals \textcite{Abdelaziz48PP100_116} aangeeft, moeten organisaties eerst een diepgaande analyse uitvoeren van de operationele workflows om specifieke vereisten te identificeren, waaronder de Recovery Point Objective (RPO) en Recovery Time Objective (RTO). Deze parameters zijn van cruciaal belang om te bepalen hoe snel een systeem kan herstellen en hoeveel dataverlies acceptabel is binnen de operationele eisen van organisaties zoals luchtvaartbedrijven.

Het implementeren van veilige en gecontroleerde synchronisatieoplossingen vereist een grondige evaluatie van bestaande technologieën, evenals een diepgaand inzicht in de risico’s die gepaard gaan met dataoverdracht in kwetsbare settings. In dit kader rijst de vraag: Welke risico’s en beperkingen zijn verbonden aan dataoverdracht en synchronisatie in kwetsbare operationele settings?
\textcite{VinayakBhuvi} benadrukken dat het gebruik van beveiligde lokale verbindingen, in combinatie met versleutelde bestanden en failover-mechanismen, een effectieve manier is om zowel data-integriteit als vertrouwelijkheid te waarborgen. Dit leidt tot de vraag: Welke technologieën en technieken kunnen bijdragen aan het waarborgen van data-integriteit en vertrouwelijkheid tijdens gegevensoverdracht in operationele omgevingen? Dit is met name relevant voor operationele omgevingen zoals EFB’s, waar data vaak lokaal moet worden gesynchroniseerd voordat deze naar een centrale opslag of cloudomgeving wordt overgebracht.

Een belangrijk aandachtspunt betreft de integratie van bestaande back-uptechnologieën, zoals Veeam Backup, in operationele settings. Dit roept de vraag op hoe deze technologieën effectief kunnen worden ingezet om te voldoen aan de specifieke eisen en uitdagingen van dergelijke omgevingen. \textcite{Yanamala2024} wijst erop dat deze tools organisaties in staat stellen om flexibele en schaalbare back-upstrategieën te ontwikkelen die zijn afgestemd op de dynamische eisen van het werkveld. Het regelmatig testen van disaster recovery-procedures, evenals het monitoren van de prestaties van synchronisatie- en back-upsystemen, is volgens de literatuur essentieel om de effectiviteit van de gekozen strategie te waarborgen.

De literatuurstudie toont aan dat een succesvolle implementatie van synchronisatie- en back-upstrategieën voor EFB's afhankelijk is van een holistische aanpak. Deze aanpak moet zowel de technologische mogelijkheden als de operationele beperkingen van het werkveld in overweging nemen. De bevindingen uit de bestaande literatuur bieden een solide basis voor het ontwikkelen van een veilige en flexibele oplossing die gegevensverlies minimaliseert en tegelijkertijd operationele continuïteit garandeert, zelfs in uitdagende omstandigheden.


%---------- Methodologie ------------------------------------------------------
\section{Methodologie}%
\label{sec:methodologie}

%Hier beschrijf je hoe je van plan bent het onderzoek te voeren. Welke onderzoekstechniek ga je toepassen om elk van je onderzoeksvragen te beantwoorden? Gebruik je hiervoor literatuurstudie, interviews met belanghebbenden (bv.~voor requirements-analyse), experimenten, simulaties, vergelijkende studie, risico-analyse, PoC, \ldots?

%Valt je onderwerp onder één van de typische soorten bachelorproeven die besproken zijn in de lessen Research Methods (bv.\ vergelijkende studie of risico-analyse)? Zorg er dan ook voor dat we duidelijk de verschillende stappen terug vinden die we verwachten in dit soort onderzoek!

%Vermijd onderzoekstechnieken die geen objectieve, meetbare resultaten kunnen opleveren. Enquêtes, bijvoorbeeld, zijn voor een bachelorproef informatica meestal \textbf{niet geschikt}. De antwoorden zijn eerder meningen dan feiten en in de praktijk blijkt het ook bijzonder moeilijk om voldoende respondenten te vinden. Studenten die een enquête willen voeren, hebben meestal ook geen goede definitie van de populatie, waardoor ook niet kan aangetoond worden dat eventuele resultaten representatief zijn.

%Uit dit onderdeel moet duidelijk naar voor komen dat je bachelorproef ook technisch voldoen\-de diepgang zal bevatten. Het zou niet kloppen als een bachelorproef informatica ook door bv.\ een student marketing zou kunnen uitgevoerd worden.

%Je beschrijft ook al welke tools (hardware, software, diensten, \ldots) je denkt hiervoor te gebruiken of te ontwikkelen.

%Probeer ook een tijdschatting te maken. Hoe lang zal je met elke fase van je onderzoek bezig zijn en wat zijn de concrete \emph{deliverables} in elke fase?

Deze thesis begint met een grondige requirementsanalyse om de functionele en niet-functionele eisen voor de synchronisatie- en back-upomgeving duidelijk in kaart te brengen. Vanuit deze analyse wordt een combinatie van literatuurstudie, simulaties en een proof-of-concept (PoC) toegepast om de centrale onderzoeksvraag te beantwoorden:

Hoe kan een veilige en gecontroleerde omgeving worden ontworpen voor de synchronisatie en back-up van data in Electronic Flight Bags zonder of met beperkte internetverbinding, om gegevensverlies te minimaliseren?

Deze aanpak zorgt ervoor dat de voorgestelde oplossing niet alleen technisch haalbaar is, maar ook voldoet aan de specifieke eisen en beperkingen van de werkomgeving waarin het toegepast zal worden.

\subsection{Fase 1: Requirementsanalyse}
De eerste fase van het onderzoek richt zich op het uitvoeren van een gedetailleerde requirementsanalyse. Deze analyse is bedoeld om de technische en operationele vereisten te identificeren die nodig zijn voor de ontwikkeling van een veilige en flexibele synchronisatie- en back-upomgeving voor Electronic Flight Bags (EFB's). Hierbij worden de parameters Recovery Time Objective (RTO), Recovery Point Objective (RPO), databeveiliging en operationele continuïteit centraal gesteld.

De requirementsanalyse omvat een grondige evaluatie van de operationele workflows en uitdagingen, waaronder de impact van beperkte connectiviteit, de aard van de te verwerken data, en de schaalbaarheid van de oplossing. De resultaten van deze analyse zullen dienen als leidraad voor de daaropvolgende fasen van het onderzoek.

Mijlpaal: Een requirementsdocument waarin de technische en operationele eisen van het proof-of-concept worden vastgelegd.

\subsection{Fase 2: Literatuurstudie}
De tweede fase van het onderzoek omvat een uitgebreide literatuurstudie, gericht op de huidige technieken en best practices met betrekking tot veilige synchronisatie- en back-upstrategieën voor Electronic Flight Bags (EFB's) in operationele contexten. Hierbij wordt academische en professionele vakliteratuur geanalyseerd om de belangrijkste parameters en uitdagingen op dit gebied te identificeren, zoals databeveiliging, Recovery Time Objective (RTO), en Recovery Point Objective (RPO). Deze literatuurstudie zal ongeveer drie weken duren en resulteren in een overzicht van relevante technologieën en strategieën die kunnen bijdragen aan het ontwerp van een flexibele en veilige oplossing.

Mijlpaal: Een rapport dat een overzicht biedt van relevante technologieën, strategieën en best practices, met een samenvatting van de belangrijkste bevindingen uit de literatuur.

\subsection{Fase 3: Simulaties en Proof-of-Concept (PoC)}
In de derde fase van het onderzoek zal een proof-of-concept worden ontwikkeld waarin een veilige en gecontroleerde synchronisatie- en back-upomgeving voor Electronic Flight Bags (EFB's) wordt gesimuleerd. Hiervoor wordt gebruikgemaakt van Veeam Backup om het back-upaspect van de oplossing te realiseren. Daarnaast zal een virtuele omgeving worden opgezet met VirtualBox om zowel de lokale infrastructuur als mogelijke cloudcomponenten te simuleren.

Het proof-of-concept zal worden getest in verschillende scenario's, zoals dataverlies, beperkte connectiviteit, en systeemuitval, om de effectiviteit van de synchronisatie- en back-upstrategie te evalueren. Tijdens deze experimenten zullen de parameters Recovery Point Objective (RPO) en Recovery Time Objective (RTO) worden gemeten om de prestaties van de oplossing te analyseren. Deze fase is gepland om ongeveer vier weken te duren, met als eindresultaat een werkend prototype dat de flexibiliteit en betrouwbaarheid van de voorgestelde strategie valideert.

Mijlpaal: Een volledig functionerend proof-of-concept dat realistische scenario’s simuleert en operationele parameters zoals RTO en RPO meet.

\subsection{Fase 4: Validatie en Evaluatie}
In de vierde fase zullen de resultaten van het PoC worden gevalideerd door middel van prestatiemetingen en analyses. De prestaties zullen worden geëvalueerd op basis van de herstelbaarheid en veiligheid van de gegevens, en er zal een vergelijking worden gemaakt met de theoretische verwachtingen uit de literatuurstudie. Python en Excel zullen worden gebruikt voor het analyseren en visualiseren van de prestatiegegevens. Deze fase neemt twee weken in beslag en resulteert in een evaluatierapport dat de effectiviteit van de oplossing samenvat.

Mijlpaal: Een evaluatierapport waarin de prestaties van het PoC worden geanalyseerd en vergeleken met de theoretische verwachtingen, inclusief aanbevelingen voor toekomstige verbeteringen.

\subsection{Tools en Technologieën}
Voor de uitvoering van deze bachelorproef worden de volgende tools en technologieën ingezet:
\begin{itemize}
    \item \textbf{VirtualBox}: voor het creëren van een virtuele on-premise server.
    \item \textbf{Veaam Backup}: voor het beheren en uitvoeren van back-upstrategieën.
    \item \textbf{Cloudopslagdienst}: als cloudoplossing voor het opslaan van back-ups.
    \item \textbf{Python / Excel}: voor data-analyse en visualisatie van de resultaten.
\end{itemize}

\subsection{Tijdsplanning en Deliverables}
Het onderzoek is verdeeld in de volgende fasen, met de bijbehorende tijdsplanning:
\begin{itemize}
    \item \textbf{Mijlpaal 1:} Requirementsanalyse
    Resultaat: Gedetailleerde lijst van functionele en niet-functionele eisen, inclusief randvoorwaarden en een duidelijk probleemkader.
    
    \item \textbf{Mijlpaal 2:} Literatuurstudie 
    Resultaat: Uitgebreid literatuuroverzicht dat theorieën, bestaande oplossingen en technologische uitdagingen behandelt. Hierbij kan voortschrijdend inzicht leiden tot aanpassingen in de eisen en doelen.  
    
    \item \textbf{Mijlpaal 3:} Proof-of-Concept  
    Resultaat: Werkend prototype van de synchronisatie- en back-upstrategie, gebaseerd op de vastgelegde eisen en kennis uit de literatuurstudie. Tijdens de ontwikkeling kunnen nieuwe inzichten leiden tot aanvullende onderzoeksvragen.  
    
    \item \textbf{Mijlpaal 4:} Validatie en Evaluatie  
    Resultaat: Evaluatierapport met een prestatieanalyse van de PoC. Inzichten uit deze fase kunnen waar nodig leiden tot iteraties in eerdere mijlpalen. 
\end{itemize}


%---------- Verwachte resultaten ----------------------------------------------
\section{Verwacht resultaat, conclusie}%
\label{sec:verwachte_resultaten}

%Hier beschrijf je welke resultaten je verwacht. Als je metingen en simulaties uitvoert, kan je hier al mock-ups maken van de grafieken samen met de verwachte conclusies. Benoem zeker al je assen en de onderdelen van de grafiek die je gaat gebruiken. Dit zorgt ervoor dat je concreet weet welk soort data je moet verzamelen en hoe je die moet meten.

%Wat heeft de doelgroep van je onderzoek aan het resultaat? Op welke manier zorgt jouw bachelorproef voor een meerwaarde?

%Hier beschrijf je wat je verwacht uit je onderzoek, met de motivatie waarom. Het is \textbf{niet} erg indien uit je onderzoek andere resultaten en conclusies vloeien dan dat je hier beschrijft: het is dan juist interessant om te onderzoeken waarom jouw hypothesen niet overeenkomen met de resultaten.

In dit onderzoek verwacht ik dat de implementatie van een veilige en gecontroleerde synchronisatie- en back-upstrategie voor Electronic Flight Bags (EFB's) zal resulteren in een oplossing die voldoet aan de eisen voor gegevensintegriteit, continuïteit, en beveiliging in diverse operationele contexten. De belangrijkste parameters die gemeten worden, zijn de \textbf{Recovery Point Objective (RPO)} en de \textbf{Recovery Time Objective (RTO)}. De RPO bepaalt hoeveel recente data kan worden teruggehaald in geval van dataverlies, terwijl de RTO aangeeft hoe snel het systeem hersteld kan worden na een incident.

Op basis van de simulaties wordt verwacht dat de RPO binnen enkele minuten tot uren kan blijven, afhankelijk van de frequentie en configuratie van de synchronisaties. De RTO zal naar verwachting variëren van enkele minuten tot een paar uur, afhankelijk van de grootte van de data en de gebruikte synchronisatie- en back-upmethoden. Verschillende scenario's zullen worden getest, waaronder data-overdracht via een lokale secure verbinding, synchronisatie via de cloud, en herstel vanuit een off-site back-up.

\subsection{Mock-up van Verwachte Resultaten}
Voor de grafieken verwacht ik bijvoorbeeld een lijn- of staafdiagram te maken waarin op de X-as de tijd (in uren) wordt weergegeven en op de Y-as de hersteltijd (RTO) in minuten of uren wordt gemeten. Elke lijn in de grafiek zal verschillende back-upmethoden (bijvoorbeeld volledig versus incrementeel) vertegenwoordigen. Dit biedt een duidelijk visueel overzicht van hoe de hersteltijd varieert per methode.

\subsection{Waarde voor de Doelgroep} De meerwaarde van dit onderzoek ligt in het bieden van concrete richtlijnen voor IT-professionals en organisaties die verantwoordelijk zijn voor het beheer van Electronic Flight Bags (EFB's) in uitdagende operationele settings. Dit onderzoek helpt bij het ontwikkelen van een veilige en flexibele synchronisatie- en back-upstrategie, waarbij wordt ingespeeld op de noodzaak om gegevensverlies en operationele verstoringen te minimaliseren. Door praktische inzichten te verschaffen over het combineren van verschillende synchronisatie- en back-upmethoden in situaties met variërende connectiviteit, ondersteunt dit onderzoek organisaties bij het waarborgen van gegevensbeveiliging en continuïteit in kritieke scenario's.

\subsection{Conclusie}
Dit onderzoek zal waardevolle inzichten opleveren over de effectiviteit van synchronisatie- en back-upstrategieën voor Electronic Flight Bags (EFB's) in operationele omgevingen met beperkte of afwisselende connectiviteit. De resultaten zullen een duidelijke evaluatie bieden van de balans tussen veiligheid, snelheid en betrouwbaarheid bij het implementeren van dergelijke systemen. Zelfs wanneer bepaalde strategieën minder optimaal blijken te werken in specifieke scenario's, zijn deze bevindingen essentieel om te begrijpen welke aspecten verbetering behoeven en waarom bepaalde oplossingen beter geschikt zijn voor specifieke contexten. Deze inzichten vormen een praktische leidraad voor organisaties die veilige en flexibele dataoplossingen willen implementeren in uitdagende operationele settings.



    
    \printbibliography[heading=bibintoc]
    
\end{document}