%==============================================================================
% Sjabloon onderzoeksvoorstel bachproef
%==============================================================================
% Gebaseerd op document class `hogent-article'
% zie <https://github.com/HoGentTIN/latex-hogent-article>

% Voor een voorstel in het Engels: voeg de documentclass-optie [english] toe.
% Let op: kan enkel na toestemming van de bachelorproefcoördinator!
\documentclass{hogent-article}

% Invoegen bibliografiebestand
\addbibresource{voorstel.bib}

% Informatie over de opleiding, het vak en soort opdracht
\studyprogramme{Professionele bachelor toegepaste informatica}
\course{Bachelorproef}
\assignmenttype{Onderzoeksvoorstel}
% Voor een voorstel in het Engels, haal de volgende 3 regels uit commentaar
% \studyprogramme{Bachelor of applied information technology}
% \course{Bachelor thesis}
% \assignmenttype{Research proposal}

\academicyear{2042-2025} 


\title{Ontwerp van een Hybride Cloud Backupstrategie: Beste Praktijken en Uitdagingen voor Gegevensherstel en Bedrijfscontinuïteit}


\author{Stein Van Driessche}
\email{stein.vandriesschen@student.hogent.be}


% Gaat het om een bachelorproef in samenwerking met een student in een andere
% opleiding? Geef dan de naam en emailadres hier
% \author{Yasmine Alaoui (naam opleiding)}
% \email{yasmine.alaoui@student.hogent.be}


\supervisor[Co-promotor]{T. De Quick (Quick IT GCV, \href{mailto:toon\_dequick@hotmail.com}{toon\_dequick@hotmail.com})}

% Binnen welke specialisatierichting uit 3TI situeert dit onderzoek zich?
% Kies uit deze lijst:
%
% - Mobile \& Enterprise development
% - AI \& Data Engineering
% - Functional \& Business Analysis
% - System \& Network Administrator
% - Mainframe Expert
% - Als het onderzoek niet past binnen een van deze domeinen specifieer je deze
%   zelf
%
\specialisation{System \& Network Administrator}
\keywords{Hybride Cloud, Backupstrategie, Gegevensherstel, Bedrijfscontinuïteit, Disaster Recovery, Recovery Point Objective (RPO), Recovery Time Objective (RTO), Cloudinfrastructuren, On-Premise}


\begin{document}
    
    \begin{abstract}
        Dit onderzoek richt zich op het ontwerpen van een hybride cloud-backupstrategie die optimale bedrijfscontinuïteit en gegevensherstel waarborgt in organisaties die gebruik maken van zowel on-premise als cloud-infrastructuren. De probleemstelling draait om de uitdagingen waarmee organisaties worden geconfronteerd bij het ontwikkelen van een effectieve strategie om gegevens te beveiligen in hybride omgevingen, waarbij downtime en gegevensverlies tot een minimum worden beperkt. De centrale onderzoeksvraag luidt: Hoe kan een hybride cloud-backupstrategie worden ontworpen om een robuuste gegevensherstel- en disaster recovery-oplossing te bieden die voldoet aan moderne prestatie- en beveiligingsvereisten? De doelstelling van het onderzoek is om een concrete, schaalbare en kostenbewuste back-upoplossing te ontwerpen en te valideren via een proof-of-concept (PoC). Dit onderzoek maakt gebruik van een literatuurstudie, aangevuld met technische experimenten waarbij verschillende hybride cloudoplossingen worden getest op parameters zoals Recovery Time Objective (RTO), Recovery Point Objective (RPO), en beveiligingsniveaus. Verwachte resultaten zijn een set van best practices voor het implementeren van een hybride cloud-backupstrategie, evenals een evaluatie van veelgebruikte tools en technieken voor disaster recovery. De meerwaarde van dit onderzoek ligt in het aanbieden van praktische inzichten en richtlijnen die IT-beheerders en bedrijven kunnen gebruiken om hun hybride cloudomgevingen te optimaliseren voor maximale gegevensbeschikbaarheid en beveiliging.
    \end{abstract}
    
    \tableofcontents
    
    % De hoofdtekst van het voorstel zit in een apart bestand, zodat het makkelijk
    % kan opgenomen worden in de bijlagen van de bachelorproef zelf.
    %---------- Inleiding ---------------------------------------------------------


% vorig jaar hebt ingediend? Heb je daarbij eventueel samengewerkt met een
% andere student?
% Zo ja, haal dan de tekst hieronder uit commentaar en pas aan.

%\paragraph{Opmerking}

% Dit voorstel is gebaseerd op het onderzoeksvoorstel dat werd geschreven in het
% kader van het vak Research Methods dat ik (vorig/dit) academiejaar heb
% uitgewerkt (met medesturent VOORNAAM NAAM als mede-auteur).
% 

\section{Inleiding}%
\label{sec:inleiding}

%Waarover zal je bachelorproef gaan? Introduceer het thema en zorg dat volgende zaken zeker duidelijk aanwezig zijn:

%\begin{itemize}
  %\item kaderen thema
  %\item de doelgroep
  %\item de probleemstelling en (centrale) onderzoeksvraag
  %\item de onderzoeksdoelstelling
%\end{itemize}

%Denk er aan: een typische bachelorproef is \textit{toegepast onderzoek}, wat betekent dat je start vanuit een concrete probleemsituatie in bedrijfscontext, een \textbf{casus}. Het is belangrijk om je onderwerp goed af te bakenen: je gaat voor die \textit{ene specifieke probleemsituatie} op zoek naar een goede oplossing, op basis van de huidige kennis in het vakgebied.

%De doelgroep moet ook concreet en duidelijk zijn, dus geen algemene of vaag gedefinieerde groepen zoals \emph{bedrijven}, \emph{developers}, \emph{Vlamingen}, enz. Je richt je in elk geval op it-professionals, een bachelorproef is geen populariserende tekst. Eén specifiek bedrijf (die te maken hebben met een concrete probleemsituatie) is dus beter dan \emph{bedrijven} in het algemeen.

%Formuleer duidelijk de onderzoeksvraag! De begeleiders lezen nog steeds te veel voorstellen waarin we geen onderzoeksvraag terugvinden.

%Schrijf ook iets over de doelstelling. Wat zie je als het concrete eindresultaat van je onderzoek, naast de uitgeschreven scriptie? Is het een proof-of-concept, een rapport met aanbevelingen, \ldots Met welk eindresultaat kan je je bachelorproef als een succes beschouwen?


%---------- Stand van zaken ---------------------------------------------------

In de context van de toenemende adoptie van cloud computing staan veel bedrijven voor de uitdaging om hun bestaande on-premise infrastructuren effectief te integreren met cloudoplossingen. Dit onderzoek richt zich op het ontwerpen van een hybride cloud-backupstrategie die zowel robuuste gegevensbeveiliging als bedrijfscontinuïteit waarborgt. De doelgroep van deze bachelorproef bestaat uit IT-professionals en systeembeheerders die verantwoordelijk zijn voor het beheer van hybride IT-infrastructuren binnen middelgrote tot grote organisaties. Deze bedrijven opereren in complexe omgevingen waarin zowel fysieke als cloudgebaseerde systemen moeten worden beschermd tegen dataverlies en onverwachte uitval, veroorzaakt door factoren zoals hardwarefalen, cyberaanvallen en natuurrampen.

De probleemstelling die centraal staat in dit onderzoek is de vraag hoe bedrijven in een hybride cloudomgeving een effectieve backupstrategie kunnen implementeren die zowel voldoet aan de moderne beveiligingsvereisten als aan de operationele eisen van minimale downtime en snelle hersteltijden. Traditionele back-upmethoden schieten vaak tekort in hybride omgevingen door de complexiteit van het beheer van meerdere locaties en de noodzaak om data consistent te synchroniseren tussen on-premise systemen en cloudopslag. Deze uitdagingen worden verder benadrukt door \textcite{Yanamala2024}, die de noodzaak voor een geïntegreerde aanpak voor hybride cloudstrategieën onderstreept. De centrale onderzoeksvraag luidt dan ook: \textit{Hoe kan een hybride cloud-backupstrategie worden ontworpen om zowel de beveiliging van gegevens als de bedrijfscontinuïteit te optimaliseren?}

De doelstelling van dit onderzoek is om een concrete en schaalbare back-upoplossing te ontwikkelen die specifiek gericht is op hybride cloudomgevingen. Het eindresultaat zal bestaan uit een proof-of-concept waarbij verschillende technologieën en tools worden getest en geëvalueerd op hun effectiviteit met betrekking tot \textit{Recovery Point Objectives} (RPO), \textit{Recovery Time Objectives} (RTO), en beveiligingseisen. Dit sluit aan bij de bevindingen van \textcite{VinayakBhuvi}, die de voordelen van cloudgebaseerde disaster recovery benadrukken. Deze bachelorproef zal, naast het proof-of-concept, ook resulteren in een rapport met aanbevelingen en best practices die IT-professionals kunnen helpen bij het verbeteren van hun back-up- en disaster recovery-plannen in hybride omgevingen.

Het onderzoek zal worden gebaseerd op een concrete casus uit het werkveld, waarin een middelgrote organisatie worstelt met het implementeren van een hybride back-upoplossing. Door de huidige technologieën te evalueren en te testen, zal dit onderzoek bijdragen aan de bestaande kennis en concrete richtlijnen bieden voor het effectief inzetten van hybride cloudoplossingen, zoals besproken door \textcite{ARASET48PP100_116}.
​
\section{Literatuurstudie}% 
\label{sec}

Hybride cloudomgevingen winnen aan populariteit onder organisaties die zowel de voordelen van on-premise infrastructuren als die van cloudoplossingen willen benutten. Deze combinatie biedt aanzienlijke voordelen, zoals verbeterde flexibiliteit, schaalbaarheid en de mogelijkheid om gegevens veilig op verschillende locaties op te slaan. Volgens \textcite{Yanamala2024} stelt het gebruik van hybride cloudstrategieën organisaties in staat om data op meerdere plekken te beheren, wat cruciaal is voor het verminderen van de kans op dataverlies en het waarborgen van de operationele continuïteit. Deze aanpak stelt bedrijven niet alleen in staat om snel te herstellen van storingen, maar verhoogt ook hun weerbaarheid tegen potentiële rampen.

Echter, de implementatie van een hybride cloud-backupstrategie gaat niet zonder uitdagingen. In het onderzoek van \textcite{ARASET48PP100_116} wordt benadrukt dat organisaties een grondige analyse van hun huidige IT-infrastructuur moeten uitvoeren voordat ze overstappen op hybride oplossingen. Het is noodzakelijk om de specifieke vereisten van de bedrijfsprocessen goed in kaart te brengen, zodat de gekozen strategie aansluit bij de operationele behoeften. Daarnaast is het regelmatig testen van disaster recovery-plannen van vitaal belang om te verzekeren dat deze strategieën effectief zijn en in lijn blijven met de bedrijfsdoelstellingen.

Een belangrijk voordeel van hybride cloudoplossingen is de elasticiteit van de cloudinfrastructuur, die bedrijven in staat stelt om hun capaciteiten aan te passen op basis van de behoeften. Volgens \textcite{VinayakBhuvi} kunnen organisaties hun infrastructuur eenvoudig opschalen wanneer dat nodig is, zonder de kosten en complexiteit die gepaard gaan met het onderhoud van een volledige duplicaat van hun productieomgeving. Dit biedt niet alleen een kostenvoordeel, maar stelt bedrijven ook in staat om hun resources efficiënter te gebruiken, wat leidt tot verbeterde prestaties en lagere operationele kosten.

Desondanks brengt de integratie van cloudtechnologieën in bestaande systemen ook aanzienlijke uitdagingen met zich mee, waaronder compatibiliteitsproblemen en verhoogde beveiligingsrisico's. Het is essentieel dat organisaties deze risico's proactief beheren door middel van het opstellen van duidelijke beleidslijnen en het ontwikkelen van robuuste strategieën voor gegevensbeveiliging en -herstel. Het onderzoek van \textcite{Yanamala2024} wijst erop dat een gedegen aanpak van deze risico's niet alleen de kans op incidenten verkleint, maar ook bijdraagt aan het vertrouwen van stakeholders in de gekozen cloudstrategieën.

In de context van deze literatuurstudie is het evident dat hybride cloud-backupstrategieën aanzienlijke voordelen bieden, mits ze zorgvuldig worden geïmplementeerd. Het succes van deze strategieën hangt af van een grondige evaluatie van de bestaande infrastructuur en een strategische aanpak die rekening houdt met de dynamische aard van de IT-omgeving. De inzichten verkregen uit de huidige literatuur vormen een waardevolle basis voor het ontwikkelen van effectieve hybride cloudoplossingen die de continuïteit van bedrijfsprocessen waarborgen en de weerbaarheid van organisaties tegen verstoringen vergroten.


%---------- Methodologie ------------------------------------------------------
\section{Methodologie}%
\label{sec:methodologie}

%Hier beschrijf je hoe je van plan bent het onderzoek te voeren. Welke onderzoekstechniek ga je toepassen om elk van je onderzoeksvragen te beantwoorden? Gebruik je hiervoor literatuurstudie, interviews met belanghebbenden (bv.~voor requirements-analyse), experimenten, simulaties, vergelijkende studie, risico-analyse, PoC, \ldots?

%Valt je onderwerp onder één van de typische soorten bachelorproeven die besproken zijn in de lessen Research Methods (bv.\ vergelijkende studie of risico-analyse)? Zorg er dan ook voor dat we duidelijk de verschillende stappen terug vinden die we verwachten in dit soort onderzoek!

%Vermijd onderzoekstechnieken die geen objectieve, meetbare resultaten kunnen opleveren. Enquêtes, bijvoorbeeld, zijn voor een bachelorproef informatica meestal \textbf{niet geschikt}. De antwoorden zijn eerder meningen dan feiten en in de praktijk blijkt het ook bijzonder moeilijk om voldoende respondenten te vinden. Studenten die een enquête willen voeren, hebben meestal ook geen goede definitie van de populatie, waardoor ook niet kan aangetoond worden dat eventuele resultaten representatief zijn.

%Uit dit onderdeel moet duidelijk naar voor komen dat je bachelorproef ook technisch voldoen\-de diepgang zal bevatten. Het zou niet kloppen als een bachelorproef informatica ook door bv.\ een student marketing zou kunnen uitgevoerd worden.

%Je beschrijft ook al welke tools (hardware, software, diensten, \ldots) je denkt hiervoor te gebruiken of te ontwikkelen.

%Probeer ook een tijdschatting te maken. Hoe lang zal je met elke fase van je onderzoek bezig zijn en wat zijn de concrete \emph{deliverables} in elke fase?

Deze thesis maakt gebruik van een combinatie van literatuurstudie, simulaties en een proof-of-concept (PoC) om de onderzoeksvraag te beantwoorden: *Hoe kan een hybride cloud-backupstrategie worden ontworpen die voldoet aan zowel de eisen voor bedrijfscontinuïteit als gegevensbeveiliging?*

\subsection{Fase 1: Literatuurstudie}
De eerste fase van het onderzoek omvat een uitgebreide literatuurstudie, gericht op de huidige technieken en best practices op het gebied van hybride cloud-back-upstrategieën en disaster recovery. In deze fase zal academische en professionele vakliteratuur worden geanalyseerd om de belangrijkste parameters en uitdagingen in dit domein te identificeren. Deze studie zal ongeveer drie weken duren en resulteert in een overzicht van relevante technologieën en strategieën.

\subsection{Fase 2: Simulaties en Proof-of-Concept (PoC)}
In de tweede fase van het onderzoek zal een proof-of-concept worden ontwikkeld waarin een hybride cloud-back-upoplossing wordt gesimuleerd. Hiervoor wordt gebruikgemaakt van Azure Backup en Azure Site Recovery om de cloudcomponent van de oplossing te realiseren. Een virtuele on-premise serveromgeving zal worden opgezet met VirtualBox, waarbij de on-premise infrastructuur wordt nagebootst.

Het proof-of-concept zal worden getest op verschillende scenario's, zoals dataverlies en systeemuitval, om de effectiviteit van de back-upstrategie te evalueren. Tijdens deze experimenten zullen de parameters Recovery Point Objective (RPO) en Recovery Time Objective (RTO) worden gemeten om de prestaties van de back-upoplossing te analyseren. Deze fase is gepland om ongeveer vier weken te duren, met als eindresultaat een werkend prototype dat de robuustheid van de back-upstrategie valideert.

\subsection{Fase 3: Validatie en Evaluatie}
In de derde fase zullen de resultaten van het PoC worden gevalideerd door middel van prestatiemetingen en analyses. De prestaties zullen worden geëvalueerd op basis van de herstelbaarheid en veiligheid van de gegevens, en er zal een vergelijking worden gemaakt met de theoretische verwachtingen uit de literatuurstudie. Python en Excel zullen worden gebruikt voor het analyseren en visualiseren van de prestatiegegevens. Deze fase neemt twee weken in beslag en resulteert in een evaluatierapport dat de effectiviteit van de oplossing samenvat.

\subsection{Tools en Technologieën}
Voor de uitvoering van deze bachelorproef worden de volgende tools en technologieën ingezet:
\begin{itemize}
    \item \textbf{VirtualBox}: voor het creëren van een virtuele on-premise server.
    \item \textbf{Azure Backup / Site Recovery}: voor het beheren en uitvoeren van back-upstrategieën.
    \item \textbf{Cloudopslagdienst}: als cloudoplossing voor het opslaan van back-ups.
    \item \textbf{Python / Excel}: voor data-analyse en visualisatie van de resultaten.
\end{itemize}

\subsection{Tijdsplanning en Deliverables}
Het onderzoek is verdeeld in de volgende fasen, met de bijbehorende tijdsplanning:
\begin{itemize}
    \item \textbf{Week 1-3}: Literatuurstudie – Resultaat: uitgebreid literatuuroverzicht.
    \item \textbf{Week 4-7}: Simulatie en Proof-of-Concept – Resultaat: werkend prototype van de back-upstrategie.
    \item \textbf{Week 8-9}: Validatie en Evaluatie – Resultaat: evaluatierapport met prestatieanalyse.
\end{itemize}


%---------- Verwachte resultaten ----------------------------------------------
\section{Verwacht resultaat, conclusie}%
\label{sec:verwachte_resultaten}

%Hier beschrijf je welke resultaten je verwacht. Als je metingen en simulaties uitvoert, kan je hier al mock-ups maken van de grafieken samen met de verwachte conclusies. Benoem zeker al je assen en de onderdelen van de grafiek die je gaat gebruiken. Dit zorgt ervoor dat je concreet weet welk soort data je moet verzamelen en hoe je die moet meten.

%Wat heeft de doelgroep van je onderzoek aan het resultaat? Op welke manier zorgt jouw bachelorproef voor een meerwaarde?

%Hier beschrijf je wat je verwacht uit je onderzoek, met de motivatie waarom. Het is \textbf{niet} erg indien uit je onderzoek andere resultaten en conclusies vloeien dan dat je hier beschrijft: het is dan juist interessant om te onderzoeken waarom jouw hypothesen niet overeenkomen met de resultaten.

In dit onderzoek verwacht ik dat de implementatie van een hybride cloud-backupstrategie zal resulteren in een oplossing die zowel voldoet aan de eisen voor bedrijfscontinuïteit als gegevensbeveiliging. De belangrijkste parameters die gemeten worden, zijn de \textbf{Recovery Point Objective (RPO)} en de \textbf{Recovery Time Objective (RTO)}. De RPO geeft aan hoe recent de back-updata moet zijn, terwijl de RTO de hersteltijd na een incident meet.

Op basis van de simulaties verwacht ik dat de RPO binnen enkele minuten tot uren kan worden gehouden, afhankelijk van de frequentie van de back-ups. De RTO zal naar verwachting variëren tussen enkele minuten en een paar uur, afhankelijk van de grootte van de data en de snelheid van het herstelproces. Ik zal hierbij verschillende back-upscenario's testen, waaronder volledig, incrementeel en differentieel back-upherstel.

\subsection{Mock-up van Verwachte Resultaten}
Voor de grafieken verwacht ik bijvoorbeeld een lijn- of staafdiagram te maken waarin op de X-as de tijd (in uren) wordt weergegeven en op de Y-as de hersteltijd (RTO) in minuten of uren wordt gemeten. Elke lijn in de grafiek zal verschillende back-upmethoden (bijvoorbeeld volledig versus incrementeel) vertegenwoordigen. Dit biedt een duidelijk visueel overzicht van hoe de hersteltijd varieert per methode.

\subsection{Waarde voor de Doelgroep}
De meerwaarde van dit onderzoek voor IT-professionals en organisaties is dat het een duidelijk beeld geeft van hoe een hybride cloud-backupstrategie kan worden geïmplementeerd met een focus op het verbeteren van bedrijfscontinuïteit en gegevensbeveiliging. Dit onderzoek biedt een praktische gids voor het implementeren van een back-upstrategie die zowel on-premise als cloudopslag combineert, en helpt bedrijven bij het minimaliseren van downtime en gegevensverlies bij storingen of rampen.

\subsection{Conclusie}
Het onderzoek zal inzicht geven in de efficiëntie van hybride cloud-back-upsystemen en de afwegingen die moeten worden gemaakt tussen kosten, snelheid en betrouwbaarheid. Zelfs als de resultaten afwijken van de verwachtingen, is dit waardevol omdat het aantoont welke factoren mogelijk nog geoptimaliseerd moeten worden en waarom bepaalde strategieën minder goed werken in specifieke omgevingen. Deze analyse biedt dus belangrijke richtlijnen voor bedrijven die soortgelijke back-upsystemen willen implementeren.



    
    \printbibliography[heading=bibintoc]
    
\end{document}